\documentclass[72pt]{article}
\usepackage[utf8]{inputenc}

\usepackage{geometry}
\geometry{a4paper}

\usepackage{fancyhdr}
\usepackage{calc}
\usepackage{multirow}
\usepackage[explicit]{titlesec}
\usepackage{graphicx}
\usepackage{xcolor}
\usepackage[hidelinks]{hyperref}

\newlength{\w}
\setlength{\w}{\textwidth / 5 - 1.18pt}

\newcommand*\getlength[1]{\number#1}

\pagestyle{fancy}
\fancyhf{}
\rhead{\textit{Grupo Scout 109 Monte Clavijo} \\}
\lhead{Campamento de Verano 2018 \\ Quintanar} %CAMPAMENTO Y LUGAR
\cfoot{\includegraphics[width=\getlength{\w}sp]{logo}
	\includegraphics[width=\getlength{\w}sp]{logo}\includegraphics[width=\getlength{\w}sp]{logo}
	\includegraphics[width=\getlength{\w}sp]{logo}\includegraphics[width=\getlength{\w}sp]{logo}}

\titleformat{\section}
  {\Large\bfseries}{\thesection}{5em}{#1}[{\titlerule}]
 
\begin{document}
\large
\fontencoding{T1}\fontfamily{lmss}\fontseries{m}\fontshape{n}\selectfont
\begin{tabular}{|p{2cm}|c|}
\hline
\multirow{5}{*}{\includegraphics[width=2cm]{logo}} & \textbf{CLAN} \\ % RAMA
\cline{2-2}
& \textbf{FIREBALL} \\ %TITULO
\cline{2-2}
& \parbox{\textwidth-4cm}{Fecha y momento del dia: \textbf{}} \\ %Fecha
& \parbox{\textwidth-4cm}{Duracion: \textbf{1.5 horas}} \\ %Duracion
& \parbox{\textwidth-4cm}{Destinatarios/as: \textbf{jovenes de 17 a 21 años}} \\ %Destinatarios
\hline
\end{tabular}

\section*{DESCRIPCI\'ON}
Una actividad de correr estilo ``bandera'', muy entretenida y apta para todas las edades.
Podemos además jugar más de una partida si vemos que se nos hace corto. 

\section*{OBJETIVOS}
\begin{itemize}
    \item Pasarlo genial
    \item Hacer ejercicio
    \item Fomentar la competitividad
\end{itemize}

\section*{DESARROLLO}
Primero, se dividirá al grupo en grupos y se les repartirá a cada uno una bandera, que se puede tratar de
una chaqueta o de una pañoleta. Cada equipo ``esconderá'' la bandera (de una forma bastante fáil de encontrar)
en una zona asignada. El objetivo del juego es conseguir las banderas de los otros equipos sin perder la propia.

Durante el juego dispondremos de los siguientes objetos, que se encontrarán repartidos por el terreno
de juego.
\begin{itemize}
    \item Escudo. Ocupa una mano y nos servirá como protección a los ataques enemigos.
    \item Espada. Ocupa dos manos y nos servirá para matar a nuestros enemigos.
    \item Fireball. Ocupa una mano y es el único objeto arrojadizo del juego.
    \item Sombrero. Ocupa dos manos y nos servirá para revivir a nuestros compañeros.
\end{itemize}
En cada momento del juego, solo podremos tener una cantidad de objetos que sumen, como mucho, dos manos.
Es decir, no podremos tener una espada y un escudo, pero sí dos fireballs o una fireball y un escudo.

Si nos golpea una espada o una fireball en alguna parte del cuerpo que no sea una espada, una fireball
o un escudo, nos sentaremos en el sitio dejando los objetos que llevabamos en las manos en el suelo. No podremos
levantarnos hasta que un compañero nos reviva dandonos un beso con el sombrero puesto.

Al comienzo del juego, los objetos estarán repartidos por la campa y con el pitido del árbitro, comenzará el
juego. Cada vez que una bandera es robada, es decir, extraida de una base enemiga y llevada a la propia, el 
árbitro detendrá el juego y el equipo que ha perdido la bandera quedará eliminado.

Si vemos que eliminando al equipo que pierde la bandera el juego es muy corto, podemos alargarlo no eliminando
al equipo y tan solo imponiendole una penalizacion o incluso reviviendo a todos sus miembros (entonces ganaría 
el primer equipo en conseguir todas las banderas y matar a todos los miembros de los otros equipos).

\section*{MATERIALES}
\begin{itemize}
    \item Churros de piscina (cortados por la mitad formarán las espadas)
    \item Harina, agua y papel de periódico (con lo que formaremos las fireball)
    \item Cartones (que recortados, nos servirán como escudos)
    \item Sombreros
    \item Un silvato (para facilitar las labores del árbitro) (opcional)
\end{itemize}


\section*{SUGERENCIAS}
\begin{itemize}
    \item Procurar que haya más o menos un objeto por persona, tratando de no sobreexcedernos en número de
        objetos, ya que perderían valor.
\end{itemize}

\section*{EVALUACION}
\begin{itemize}
    \item ¿Hemos hecho ejercicio de una forma divertida?
    \item ¿Hemos seguido las normas del juego?
    \item ¿Nos lo hemos pasado bien?
\end{itemize}

\section*{FUENTE}
G. S. 109 Monte Clavijo
\end{document}
