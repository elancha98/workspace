\documentclass{article}
\usepackage[utf8]{inputenc}

\usepackage{geometry}
\geometry{a4paper}

\usepackage{mathtools}
\usepackage{amsfonts}

\newcommand{\abs}[1]{\left\lvert #1 \right\rvert}
\newcommand{\R}{\mathbb{R}}

\begin{document}
\title{FACULTAD DE MATEMATICAS Y ESTADISTICA \\
Calculo en una Variable. Curso 2016-2017 \\
\Large \textbf{Resolucion del examen final extraordinario del 6 de julio de 2017} }
\date{}
\maketitle

\section*{Enunciado}
\begin{enumerate}
\item Teoria
\begin{enumerate}
\item Enunciad y demostrad el teorema de Bolzano sobre ceros de funciones continuas.
\item  Dada una funcion continua $f: \mathbb{R} \longrightarrow \mathbb{R}$, consideramos
la funcion
\[
G(x) = \int_x^{x^2} f(t)dt
\]
Razonad que la funcion $G$ es derivable y probad que existe $c \in (0,1)$ con $G'(c)=0$.
\item Sea $I \subset \mathbb{R}$ un intervalo abierto y $f:I \longrightarrow \mathbb{R}$
una funcion tal que existe un numero ral $r > 1$ con $\abs{f(x) - f(y)} < \abs{x - y}^r$ para
cualquier $x, y \in I$. Probad que $f$ es derivable. ¿Que mas se puede decir sobre esta funcion?
\end{enumerate}
\item \begin{enumerate}
\item Sea $f$ una funcion derivable en el punto $a \in \R$. Determinad el limite de la sucesion
de termino general
\[
n \left( f\left( a + \frac{2}{n} \right) - 3f\left(a +\frac{1}{n} \right) + 2f(a)\right)
\]
\item Calculad, usando polinomios de Taylor, el valor de la integral
$\displaystyle \int_0^{1/2} \frac{\sin x}{x} dx$ con un error inferior a $10^{-3}$,
\end{enumerate}
\item Conderamos las funciones
\[
f(x) = \frac{\cos x}{1 + \sin^2 x}, \quad \quad g(x)=1-\frac{2x}{\pi}
\]
\begin{enumerate}
\item Determinad el dominio, los extremos y el recorrido de la funcion $f$.
\item Probad que existe $	\varepsilon > 0$ tal que $f(x) > g(x)$ si $x \in (0, \varepsilon)$ y,
en cambio, $f(x) < g(x)$ si $x \in (\pi/2-\varepsilon, \pi/2)$.
\item Probad que existe $c \in (0,\pi/2)$ con $f(c) =g(c)$
\item Probad que $\displaystyle \int_0^{\pi/2}f = \int_0^{pi/2}g$.
\item Probad el resultado del apartado (c) a partir de la igualdad del apartado (d).
\item Determinad $f^{(k)}(0)$ para $k=4$ y para $k=2017$.
\end{enumerate}
\end{enumerate}
\section*{Solucion}
\begin{enumerate}
\item \begin{enumerate}
\item Mirar teoria.
\item Vemos que 
\[
G(x) = \int_x^{0}f(t)dt + \int_0^{x^2}f(t)dt = \int_0^{x^2}f(t)dt - \int_0^x f(t)dt
\]
Consideramos ahora $\displaystyle F(x) = \int_0^x f(t)dt$, que es derivable por el teorema
fundamental del Calculo. Y, por lo tanto, $G(x) = F(x^2) - F(x)$ es derivable por ser composicion
y suma de funciones derivables. Consideramos ahora $G(0) = 0$ y $G(1) = 0$ (Como se puede
observar de la definicion de $G$). Aplicando el teorema del valor medio, obtenemos
que existe $c \in (0,1)$ tal que
\[
G'(c) = \frac{G(1)-G(0)}{1-0}=0
\]
\item Para cualquier $c \in I$, consideramos
\[
\abs{\lim_{h \to 0} \frac{f(c+h) - f(c)}{h}} = \lim_{h \to 0} \frac{\abs{f(c+h) - f(c)}}{\abs{h}}
\leq \lim_{h \to 0} \frac{\abs{c+h-c}^r}{\abs{h}} = \lim_{h \to 0}\abs{h}^{r-1} = 0
\]
Ya que $r > 1$. Por lo tanto, $f$ es derivable en todo $I$ y ademas $f'(x)=0$ $\forall x \in I$, es
decir, $f$ es constante.
\end{enumerate}
\item \begin{enumerate}
\item Consideramos 
\[
\lim_{n \to \infty} \frac{f\left( a+\frac{2}{n} \right) - 3f \left( a + \frac{1}{n} \right) + 2f(a)}{1/n} =
2\lim_{n \to \infty} \frac{f \left( a+\frac{2}{n}\right) - f(a)}{2/n} -
3\lim_{n \to \infty} \frac{f\left(a+\frac{1}{n}\right) - f(a)}{1/n}
\]
Como $f$ es derivable en $a$, el limite de la sucesion es $2f'(a) - 3f'(a) = -f'(a)$.
\item Consideramos la funcion $\displaystyle F(x) = \int_0^{x} \frac{\sin t}{t}dt$. Y calculamos su
desarrollo en serie de Taylor. Para ello, primero calculamos el desarrollo en serie de
$f(x)=\frac{\sin x}{x}$:
\[
f(x) = \frac{1}{x} \left(  x - \frac{x^3}{6} + \frac{x^5}{120} + \cdots \right)
\quad \quad
f(x) = 1 - \frac{x^2}{6} + \frac{x^4}{120} + \cdots
\]
Entonces ahora:
\[
F(x) = \int_0^{x} f(t)dt = x - \frac{x^3}{18} + \frac{x^5}{600} + \cdots
\]
Por lo tanto $F(1/2) = \frac{29203}{57600} + o(x^9)$, y podemos aproximar el valor de la integral
por $\displaystyle F(1/2) = \int_0^{1/2}\frac{\sin t}{t}dt \approx \frac{29203}{57600}$
\end{enumerate}
\item \begin{enumerate}
\item \begin{description}
\item [Dominio] Todo $\R$ (Ya que el denominador es siempre distinto de 0)
\item [Extremos] Primero calculamos la derivada
$\displaystyle f'(x) = \frac{-(\sin^2x+1)\sin x - 2\sin x \cos^2x}{(\sin^2 x + 1)^2}$ que se anula
si y solo si 
\[
-\sin^3x - \sin x - 2\sin x \cos^2x = 0 \iff
\begin{cases}
x = n\pi \quad (n \in \mathbb{Z}) \\
x = n\pi - \frac{1}{2}cos^{-1}(-5) \quad (n \in \mathbb{Z}) \\
x = n\pi + \frac{1}{2}cos^{-1}(-5) \quad (n \in \mathbb{Z})
\end{cases}
\]
Si $x = n\pi \pm \frac{1}{2}cos^{-1}(-5)$ entonces encontramos puntos de inflexion. Si $x = 2n\pi$,
entonces encontramos maximos, cuyo valor es $f(2n\pi) = 1$ y si $x = (2n+1)\pi$ entonces
encontramos minimos, cuyo valor es $f((2n+1)\pi) = -1$.
\item [Recorrido] $[-1,1]$. Ya que la funcion es $2\pi$-periodica (ya que tambien lo son $\sin$ y
$\cos$) y entre $[-\pi,\pi]$ se alcanza el maximo en $0$ con $f(0) = 1$ y el minimo en $\pi$ con
$f(\pi) = -1$, como ademas $f$ es continua, recorre todos los valores entre $-1$ y $1$. 
\end{description}
\item Consideramos $f'(0) = 0$ y $g'(0) = -2/\pi$, y por lo tanto, $g$ decrece mas rapido que $f$,
lo cual nos garantiza que existe $\varepsilon_1$ tal que $g(x) < f(x)$ en $(0,\varepsilon_1)$. De
la misma manera, $f'(\pi/2) = -1/2$ y $g'(\pi/2) = -2/\pi$, por lo tanto, (siguiendo el razonamiento
anterior), existe $\varepsilon_2$ tal que $g(x)>f(x)$ $\forall x \in (\pi/2-\varepsilon_2, \pi/2)$.
Ahora resta tomar $\varepsilon = \min \{\varepsilon_1, \varepsilon_2 \}$.
\item Tomando el resultado del apartado anterior, definimos la funcion $h(x) = f(x) - g(x)$.
Consideramos $h\left( \frac{\varepsilon}{2} \right)$ (el $\varepsilon$ calculado en el apartado
anterior) que es mayor a 0 por el apartado anterior, y $h \left(\frac{\pi-\varepsilon}{2} \right)$
que es menor que 0. Aplicando ahora Bolzano, existe
$c \in \left( \frac{\varepsilon}{2}, \frac{\pi - \varepsilon}{2} \right) \subset (0,1)$ tal que $h(c)=0$,
es decir, $f(c) = g(c)$.
\item 
\end{enumerate}
\end{enumerate}
\end{document}