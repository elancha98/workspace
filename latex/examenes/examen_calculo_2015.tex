\documentclass[DIN, pagenumber=false, fontsize=11pt, parskip=half]{scrartcl}

\usepackage[utf8]{inputenc}
\usepackage[T1]{fontenc}

\usepackage{amsmath}
\usepackage{amsfonts}

\setlength{\parindent}{0em}

\setkomafont{section}{\normalfont\bfseries\Large}

\newcommand\norm[1]{\left\lVert#1\right\rVert}
\newcommand\pdv[2]{\frac{\partial#1}{\partial#2}}

\newcommand{\mytitle}[1]{{\center\Huge\textbf{#1}}}
\newcounter{apartado}
\newcommand{\myapartado}[0]{\stepcounter{apartado}\alph{apartado})}
%\newenvironment{apartado}{
%    \stepcounter{apartado}
%    \alph{apartado}) 
%}{}
\newcommand{\mysection}[1]{\setcounter{apartado}{0}\textbf{\section*{#1}}}

%===================================
\begin{document}
\mytitle{\hfill Final cálculo diferencial 2015 \hfill}


%===================================
\mysection{Ejercicio 1}
a) $ f(0) = f(0x) = 0f(x) = 0$
\\ \\
b) La esfera de radio unidad (K) esta trivialmente acotada, por ejemplo por $B_2(0,0)$ y es cerrada
ya que 
\[
\forall \{ x^k \} \subset K \mid \{x^k\} \to q \implies \left\{ \norm{x^k} \right\} \to \norm{q}
\iff
1 = \norm{q} \implies q \in K \implies K \text{ cerrado}
\]
Y como $K$ es cerrado y acotado, $K$ es compacto. Como $f$ es continua (de clase $C^\infty$),
por el teorema de Weierstrass, $f$ toma valores extremos en $K$, en particular, denotaremos como
$p$ al valor minimo de $f$ en $K$
\\ \\
c) Tomamos $m = f(p)$ entonces:
\[
f\left( \frac{x}{\norm{x}} \right) = \frac{f(x)}{\norm{x}^2}
\implies f(x) = \norm{x}^2f \left( \frac{x}{\norm{x}} \right) \geq \norm{x}^2f(p) = \norm{x}^2m
\]
\\ \\
e) $f$ es diferenciable en el origen si (y solo si):
\[
0 = \lim_{h \to 0} \frac{f(h) - f(0) - Df(0)(h)}{\norm{h}}
\iff
\lim_{h \to 0} \frac{f(h) - Df(0)(h)}{\norm{h}} = 0
\]
Pero
 \[
 \lim_{h \to 0} \frac{f(h)}{\norm{h}} = \lim_{h \to 0} f\left( \frac{h}{\sqrt{\norm{h}}} \right)
 =
 \lim_{h \to 0} f\left( \frac{h}{\norm{h}} \frac{\norm{h}}{\sqrt{\norm{h}}} \right)
 =
 \]
 \[
 =
  \lim_{h \to 0} f\left(\sqrt{\norm{h}} \frac{h}{\norm{h}}  \right)
  =
  f \left( \lim_{h \to 0} \sqrt{\norm{h}} \frac{h}{\norm{h}} \right)
  =
  f(0) = 0
 \]
 Por lo tanto, tomando $Df(0) = 0$, tenemos que:
 \[
 0 = \lim_{h \to 0} \frac{f(h) - f(0) - Df(0)(h)}{\norm{h}}
 \]
 Y $f$ es diferenciable en el origen y $Df(0) = 0$
 \\ \\
 d) Las derivadas parciales en el origen son $D_{e_i}f(0) = Df(0)e_i = 0 \quad \forall i = 1,\cdots,n$
 
\newpage
 %===============================
\mysection{Ejercicio 2}
a) Calculamos $Jf(0,1,-1)$: \[
Jf(x, y, z) = 
\begin{pmatrix}
2xy + e^x & x^2 & 1
\end{pmatrix}
\implies
Jf(0,1,-1) =
\begin{pmatrix}
1 & 0 & 1
\end{pmatrix}
\]
Observamos que $det(\frac{\partial f}{\partial x}) \neq 0$ y que $f$ es de clase $C^\infty$, por lo tanto,
por el teorema de la funcion implicita:
\[
\begin{aligned}
	\exists  ! g \colon A \subset \mathbb{R}^2 &\to \mathbb{R} \\
	(y, z) &\mapsto g(y, z) = x
\end{aligned}
\quad \text{tal que}
\quad
f(g(y, z), y, z) = 0 \quad \forall (y, z) \in A \subset \mathbb{R}^2
\]
Calculamos ahora $P_2(g, (1,-1))$ que es:
\[
P_2(g, a)(\phi) = g(a) + Jg(a)(\phi - a) + \frac{1}{2}(\phi - a)^tHg(a)(\phi-a) \quad a = (1,-1)
\]
Calculamos ahora $Jg(a)$:
\[
Jg(a) = -\left(\pdv{f}{x}(p) \right)^{-1} \begin{pmatrix} \pdv{f}{y}(p) & \pdv{f}{z}(p) \end{pmatrix}
=
-(1)^{-1} \begin{pmatrix} 0 & 1 \end{pmatrix} = \begin{pmatrix} 0 & -1 \end{pmatrix}
\]
Calculamos ahora $Hg(a)$. Para ello derivamos implicitamente. Sabemos que $f(g(y, z), y, z) = 0$
y por lo tanto (Llamando $g(y, z) = x$):
\begin{equation} \label{eq:p}
0 = \pdv{f}{y} = x^2 + 2xy\pdv{x}{y} + e^x\pdv{x}{y}
\iff
x^2 + \pdv{x}{y}\left( 2xy + e^x \right) = 0
\end{equation}
Calculamos ahora $\pdv{^2x}{y \partial y}$, repitiendo el mismo proceso:
\[
0 = 2x\pdv{x}{y} +
\pdv{^2x}{y \partial y} \left( 2xy + e^x \right) + \pdv{x}{y} \left( 2y\pdv{x}{y} + 2x + e^x \pdv{x}{y} \right)
\]
De donde se obtiene (sustituyendo):
\[
\pdv{^2x}{y \partial y}(1,-1) = 0
\]
Ahora derivando parcialmente respecto a $z$ en la ecuacion~\eqref{eq:p} tenemos:
 \[
 0 = 2x\pdv{x}{z} + \pdv{^2x}{z \partial y} \left( 2xy +e^x \right) +
 \pdv{x}{y} \left( 2y \pdv{x}{z} + e^x \pdv{x}{z} \right)
 \implies
 \pdv{^2x}{z \partial y}(1,-1) = 0
 \]
 Como $f $ es de clase $C^\infty$, entonces $\pdv{^2x}{z\partial y} = \pdv{^2x}{y \partial z} = 0$. 
 Repetimos ahora el proceso para obtener $\pdv{^2x}{z \partial z}$:
 \[
 0 = 2xy \pdv{x}{z} + e^x\pdv{x}{z} + 1
 \iff
 0 = \pdv{x}{z} \left(2xy + e^x \right) +1
 \]
 \[
 \pdv{^2x}{z \partial z} \left( 2xy + e^x \right) + \pdv{x}{z} \left( 2y\pdv{x}{z} + e^x \pdv{x}{z} \right) = 0
 \implies
 \pdv{^2x}{z \partial z}(1,-1) = -3
 \]
Y $Hg(a) = \begin{pmatrix} 0 & 0 \\ 0& -3 \end{pmatrix}$. Para concluir:
\[
P_2(g, a)(y, z) = 0+ \begin{pmatrix} 0 & -1 \end{pmatrix}
\left( \begin{pmatrix}y \\ z \end{pmatrix} - \begin{pmatrix} 1 \\ -1 \end{pmatrix} \right) + 
\frac{1}{2}\left( \begin{pmatrix} y & z \end{pmatrix} - \begin{pmatrix} 1 & -1 \end{pmatrix} \right)
\begin{pmatrix} 0 & 0 \\ 0 & -3 \end{pmatrix}
\left( \begin{pmatrix} y \\ z \end{pmatrix} - \begin{pmatrix} 1 \\ -1 \end{pmatrix} \right)
\]
\[
P_2(g, a)(y, z) = -3z^2 -4z+1
\]
\\ \\ \\
b) Sabemos que $J(f + g) = Jf +Jg$ (si $f$ y $g$ son diferenciables) y por lo tanto,
si $h(y, z) = g(y, z) + t(y, z)$ (con $t(y, z) = -y^2 + 2y + z$) entonces $Jh(y, z) = Jg(y, z) + Jt(y, z)$
ya que tanto $t$ como $g$ son de clase $C^\infty$ y
\[
Jh(1, -1) = \begin{pmatrix} 0 & -1 \end{pmatrix} + \begin{pmatrix} 0 & 1\end{pmatrix}
= \begin{pmatrix} 0 & 0 \end{pmatrix}
\]
Por lo tanto $(1,-1)$ es punto critico. Vamos ahora a clasificarlo, para ello
calulamos $Hh(1, -1)$, teniendo en cuenta que $Hh(y, z) = Hg(y, z) + Ht(y, z)$ (de nuevo ya que
$g$ y $t$ son de clase $C^\infty$):
\[
Ht(y, z) = \begin{pmatrix} -2 & 0 \\ 0 & 0 \end{pmatrix}
\implies
Hh(1,-1) = \begin{pmatrix}0 & 0 \\ 0 & -3 \end{pmatrix} + \begin{pmatrix} -2 & 0 \\ 0 & 0 \end{pmatrix}
= \begin{pmatrix} -2 & 0 \\ 0 & -3 \end{pmatrix}
\]
Y por lo tanto $(1,-1)$ es un maximo, ya que $Hh(1,-1)$ es definida negativa (todos los
$vap$s son negativos)
%=================
 \newpage
 \mysection{Ejercicio 3}
 a) Primero calculamos $Jf(x, y)$:
 \[
 Jf(x, y) = \begin{pmatrix}
 -y^2e^{y-x^2}2x &
 2ye^{y-x^2} + y^2e^{y-x^2}
 \end{pmatrix}
 \]
 De la primera componente observamos que vale 0 $\iff \begin{cases} x =0 \\ y = 0 \end{cases}$
 \\
 La segunda, vale 0 $\iff \begin{cases} y = 0 \\ y = -2 \end{cases}$
 \\
 De donde obtenemos que los puntos criticos son $(0,-2)$ y $(x, 0) \quad \forall x \in \mathbb{R}$. Calculamos ahora $Hf(x, y)$:
 \[
 Hf(x, y) =
 \begin{pmatrix}
 4y^2x^2e^{y-x^2} - 2y^2e^{y-x^2} & -4xye^{y-x^2} - 2xy^2e^{y-x^2} \\
 -4xye^{y-x^2} -2xy^2e^{y-x^2} & 2e^{y-x^2} + 4ye^{y-x^2} + y^2e^{y-x^2}
 \end{pmatrix}
\]
\[
Hf(0,-2) = \begin{pmatrix}
-8e^{-2} & 0 \\
0 & -2e^{-2}
\end{pmatrix}
\]
Y por lo tanto $f(0,-2)$ es maximo, ya que es definda negativa porque todos los $vap$s son negativos
\[
Hf(x, 0) =
\begin{pmatrix}
0 & 0 \\
0 & 2e^{-x^2}
\end{pmatrix}
\quad \text{que es semidefinida positiva}
 \]
 Para caracterizar los puntos $(x, 0)$ observamos que $f(x, y) \geq 0 \quad \forall (x, y) \in \mathbb{R}^2 $
 y que $f(x, 0) = 0 \quad \forall x \in \mathbb{R}^2$ y por lo tanto son minimos
\\ \\ \\ \\
b) Los puntos criticos de $f$ en $K$ son $(0,0)$. Miramos ahora los puntos de la frontera $x = 0$.
Para ello definimos la funcion:
\[
F(x, y, \lambda) = y^2e^{y-x^2} + \lambda x
\]
\[
JF(x, y, \lambda) =
\begin{pmatrix} -2xy^2e^{y-x^2} + \lambda & 2ye^{y-x^2} + y^2e^{y-x^2} & x \end{pmatrix}
\]
\[
JF(x, y, \lambda) = 0 \iff
\begin{cases}
-2xy^2e^{y-x^2} + \lambda = 0 \\
 2ye^{y-x^2} + y^2e^{y-x^2} = 0 \\
 x = 0
\end{cases}
\implies
x = 0, \lambda = 0, y = \begin{cases} -2 \\ 0 \end{cases}
\]
Ahora la frontera $y = 1$:
\[
F(x, y, \lambda) = y^2e^{y-x^2} + \lambda (y-1)
\implies
JF(x, y, \lambda) =
\begin{pmatrix} -2xy^2e^{y-x^2} & 2ye^{y-x^2} + y^2e^{y-x^2} + \lambda & y - 1 \end{pmatrix}
\]
\[
JF(x, y, \lambda) = 0 \iff x = 0, y = 1, \lambda = -3e
\]
Con la frontera $x^2 = y$:
\[
F(x, y, \lambda) = y^2e^{y-x^2} + \lambda (x^2 - y)
\implies
JF(x, y, \lambda) =
\begin{pmatrix} -2xy^2e^{y-x^2} + 2\lambda x & 2ye^{y-x^2} + y^2e^{y-x^2} - \lambda &
x^2 - y \end{pmatrix}
\]
\[
JF(x, y, \lambda) = 0 \iff x = 0, y = 0, \lambda = 0
\]
Ahora con la restriccion $x=0$ y $y=1$:
\[
F(x, y, \lambda_1, \lambda_2) = y^2e^{y-x^2} + \lambda_1 x + \lambda_2 (y-1) \implies
\]
\[
\implies JF(x, y, \lambda_1, \lambda_2) = 
\begin{pmatrix}
-2xy^2e^{y-x^2} + \lambda_1 & 2ye^{y-x^2} + y^2e^{y-x^2} + \lambda_2 & x & y
\end{pmatrix}
\]
\[
JF(x, y, \lambda_1, \lambda_2) = 0 \iff x=0,y=0,\lambda_1=0,\lambda_2=0
\]
Ahora con la restriccion $x=0$ y $x^2=y$:
\[
F(x, y, \lambda_1, \lambda_2) = y^2e^{y-x^2} + \lambda_1 x + \lambda_2 (x^2 - y) \implies
\]
\[
\implies JF(x, y, \lambda_1, \lambda_2) = 
\begin{pmatrix}
-2xy^2e^{y-x^2} + \lambda_1 + 2\lambda_2x & 2ye^{y-x^2} + y^2e^{y-x^2} - \lambda_2 & x & x^2 - y
\end{pmatrix}
\]
\[
JF(x, y, \lambda_1, \lambda_2) = 0 \iff x=0,y=0,\lambda_1=0,\lambda_2=0
\]
Ahora con $y=1$ y $x^2=y$:
\[
F(x, y, \lambda_1, \lambda_2) = y^2e^{y-x^2} + \lambda_1 (y-1) + \lambda_2 (x^2 - y) \implies
\]
\[
\implies JF(x, y, \lambda_1, \lambda_2) = 
\begin{pmatrix}
-2xy^2e^{y-x^2} + 2\lambda_2x & 2ye^{y-x^2} + y^2e^{y-x^2} +\lambda_1 - \lambda_2 & y-1 & x^2 - y
\end{pmatrix}
\]
\[
JF(x, y, \lambda_1, \lambda_2) = 0 \iff x=\pm 1, y =1, \lambda_1 = 1-3e, \lambda_2=1
\]
Por ultimo, tomamos como restriccion $x=0$, $y=1$, $x^2=y$:
\[
F(x, y, \lambda_1, \lambda_2, \lambda_3) = y^2e^{y-x^2} + \lambda_1 x + \lambda_2 (y-1) +
\lambda_3 (x^2 - y) \implies
\]
\[
\implies JF(x, y, \lambda_1, \lambda_2, \lambda_3) = 
\begin{pmatrix}
-2xy^2e^{y-x^2} + \lambda_1 + 2\lambda_2 x & 2ye^{y-x^2} + y^2e^{y-x^2} + \lambda_2 - \lambda_3
& x & y-1 & x^2 - y
\end{pmatrix}
\]
\[
JF(x, y, \lambda_1, \lambda_2, \lambda_3) \neq 0 \quad
\forall (x, y, \lambda_1, \lambda_2, \lambda_3) \in \mathbb{R}^5
\]
Los candidatos a maximo y aminimo son:
\[
\begin{cases}
(0,0) \\
(0,1) \\
(1, 1)
\end{cases}
\implies
\begin{cases}
f(0, 0) = 0 \\
f(0, 1) = e \\
f(1, 1) = 1
\end{cases}
\implies
\begin{cases}
\text{maximo: } (0, 1) \\
\text{minimo: } (0, 0)
\end{cases}
\]
\end{document}