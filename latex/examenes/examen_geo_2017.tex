\documentclass{article}

\usepackage[utf8]{inputenc}

\usepackage{geometry}
\geometry{a4paper}

\usepackage{enumitem}

\usepackage{tikz}
\tikzstyle{every node}=[circle, draw, fill=black!50, inner sep=0pt, minimum width=4pt]

\usepackage{amsmath}
\usepackage{amssymb}
\usepackage{amsfonts}

\newcommand{\R}{\mathbb{R}}
\newcommand{\vv}[1]{\overrightarrow{#1}}
\newcommand{\norm}[1]{\lVert #1 \rVert }


\begin{document}

\title{Examen final geometria 2017}
\date {}
\maketitle

\begin{enumerate}
\item Tomamos la referencia $\mathcal{R} = \left\{D,\vv{DP},\vv{PC}\right\}$, entonces
\begin{itemize}
\item $D = \begin{pmatrix} 0 & 0 \end{pmatrix}$
\item $P = \begin{pmatrix} 1 & 0 \end{pmatrix}$
\item $A = \begin{pmatrix} 1 & -1 \end{pmatrix}$
\item $C = \begin{pmatrix} 1 & 1 \end{pmatrix}$
\end{itemize}
Ahora $\vv{DB} = \frac{3}{2} \vv{DP} \implies B = \begin{pmatrix} \frac{3}{2} & 0 \end{pmatrix}$.
Tambien $\vv{BC} = \frac{4}{3}\vv{BY} \iff \vv{BY} = \frac{3}{4}\vv{BC} = \frac{3}{4}
\begin{pmatrix} -\frac{1}{2} & 1 \end{pmatrix} = \begin{pmatrix} -\frac{3}{8} & \frac{3}{4} \end{pmatrix}$.
Por lo tanto $Y = \begin{pmatrix} \frac{9}{8} & \frac{3}{4} \end{pmatrix}$.
\\ \\
Ahora $\vv{PX} = \frac{1}{3}\vv{AC} = \frac{1}{3}\begin{pmatrix} 0 & 2 \end{pmatrix} = 
\begin{pmatrix} 0 & \frac{2}{3} \end{pmatrix} \implies X = \begin{pmatrix} 1 & \frac{2}{3} \end{pmatrix}$
\\ \\
Y $\frac{9}{8} \vv{DX} = \frac{9}{8}\begin{pmatrix} 1 & \frac{2}{3} \end{pmatrix} =
\begin{pmatrix} \frac{9}{8} & \frac{3}{4} \end{pmatrix} = \vv{DY}$ por lo tanto $D,X,Y$ estan alineados.
\\ \\ \\
\item Primero calculamos la familia de puntos fijos:
\[
L :=
\begin{cases}
4x+9y-3=x \\
-x-2y+1=y
\end{cases}
\implies
x + 3y = 1
\iff
\begin{cases}
x = -3t +1 \\
y = t
\end{cases}
\]
Que es una recta. Tomamos un centro de nuestra referencia un $P \in L$, por ejemplo
$P = \begin{pmatrix} 1 & 0 \end{pmatrix}$. Calculamos ahora $Q_f(t)$:
\[
Q_f(t) = \det \begin{pmatrix} 4-t & 9 \\ -1 & -2-t \end{pmatrix} = t^2 -2t + 1 = (t-1)^2
\]
Como $\ker (f-Id) \neq \R^2$, $f$ no diagonaliza. Calculamos ahora $u_1$. Elegimos un punto
$Q \notin L (\text{recta de puntos fijos})$,
tal que $f(Q)-Q = (f-Id)Q = u_1$ y $u_1$ sea $vep$ de $vap$ 1:
\[
\begin{pmatrix} 3 & 9 & -3 \\ -1 & -3 &1 \\ 0 & 0 & 0\end{pmatrix}
\begin{pmatrix} x \\ y \\ 1 \end{pmatrix} = \lambda \begin{pmatrix} -3 \\ 1 \end{pmatrix}
\]
Ya que si $u_1$ es $vep$ de $vap$ 1, entonces es un multiplo del vector director de $L$.
Podemos tomar, por ejemplo,
$Q = \begin{pmatrix} 1 & 1 \end{pmatrix} \implies u_1 = \begin{pmatrix} 9 & -3 \end{pmatrix}$
y $u_2 = Q - P = \begin{pmatrix} 0 & 1 \end{pmatrix} \implies \mathcal{R} =
\left\{P,u_1,u_2\right\} = \left\{\begin{pmatrix} 1 & 0 \end{pmatrix}, \begin{pmatrix} 9 & -3 \end{pmatrix},
\begin{pmatrix} 0 & 1 \end{pmatrix} \right\}$ y:
\[
M_{\mathcal{R}}(f) = \begin{pmatrix} 1 & 1 & 0 \\ 0 & 1 & 0 \\ 0 & 0 & 1 \end{pmatrix}
\]
$f$ es por tanto, una homologia general de eje $L$.
\\ \\ \\ \\
\item Primero calculamos $P_p(Q)$. $q \in P_p(Q) \iff \phi(p,q) = 0$:
\[
\phi(p,q) = \begin{pmatrix} 2 & 0 & 1 \end{pmatrix}
\begin{pmatrix} 1 & -1 & 0 \\ -1 & 3 & 1 \\ 0 & 1 & 0 \end{pmatrix}
\begin{pmatrix} x \\ y \\ 1 \end{pmatrix} = 0
\iff
\begin{cases}
x = \lambda \\
y = 2 \lambda
\end{cases}
\iff
\begin{pmatrix} x \\ y \end{pmatrix} = \lambda \begin{pmatrix} 1 \\ 2 \end{pmatrix} 
\]
Llamamos ahora $A=\begin{pmatrix} 1 & 0 \end{pmatrix}$ y $B = \begin{pmatrix} 0 & 1 \end{pmatrix}$,
dado un punto $q$, tenemos que:
\[
area(\overset{\triangle}{ABq}) = \frac{1}{2}\sqrt{G \left( \vv{AB},\vv{Aq} \right)}
=
\frac{1}{2}\norm{\vv{Aq}'}\sqrt{G \left( \vv{AB} \right)}
=
\frac{\norm{\vv{Aq}'} \norm{\vv{AB}}}{2}
\]
Donde $\vv{Aq}'$ es la proyecion de $\vv{Aq}$ sobre $\left[\vv{AB} \right]^{\perp}$. Tenemos que
$\left[\vv{AB} \right] = \left[ \begin{pmatrix} -1 & 1 \end{pmatrix}\right]$ y que
$\left[\vv{AB} \right]^{\perp} = \left[ \begin{pmatrix} 1 & 1 \end{pmatrix}\right]^{\perp}$, por lo tanto:
\[
\vv{Aq} = \begin{pmatrix} -1 \\ 0 \end{pmatrix} + \lambda \begin{pmatrix} 1 \\ 2 \end{pmatrix}
=
a \begin{pmatrix} -1 \\ 1 \end{pmatrix} + b  \begin{pmatrix} 1 \\ 1 \end{pmatrix}
\iff
b = \frac{3 \lambda - 1}{2}
\]
Entonces $\vv{Aq}' = b \begin{pmatrix}1 & 1 \end{pmatrix}$. Ahora queremos que
$area(\triangle ABq) = 1$, lo cual implica que $\norm{\vv{Aq}'} = \sqrt{2}$
\[
\norm{\vv{Aq}'} = \sqrt{2 \left( b^2 \right)} = \sqrt{2}
\iff
\frac{3 \lambda - 1}{2} = \pm 1
\]
De donde deducimos que los puntos son:
\[
\begin{cases}
q_1 = \frac{-1}{3} \begin{pmatrix} 1 & 2\end{pmatrix} =
\begin{pmatrix} \frac{-1}{3} & \frac{-2}{3}\end{pmatrix} \\
q_2 =  \begin{pmatrix} 1 & 2 \end{pmatrix}
\end{cases}
\]
\\ \\ \\ \\
\end{enumerate}
\begin{enumerate}[label=(\Alph*)]
\item Primero, vemos que $f = S_v \circ S_u$ es una rotacion. $S_v$ y $S_u$ son ambos movimientos
directos, por lo tanto, $f$ es tambien un movimiento directo, es decir, $f$ es una traslacion, una rotacion
o un movimiento helicoidal. Vemos ahora que $f(P) = P$ (ya que es punto fijo tanto por $S_v$ como por
$S_u$) es un punto fijo, lo cual implica que $f$ es una rotacion. Vemos ahora que $P + [w]$ es una
recta de puntos fijos. Vemos que $\forall x \in P +[w] \quad S_u(x) = x - 2\vv{Px}$ y
$S_v(x) = x - 2\vv{Px}$. Por lo tanto $(S_v \circ S_u)(x) = x$. Para calcular el angulo, miramos un punto
$q \in P + [u]$, entonces $f(q) = S_v(q)$, y vemos que el angulo $\alpha$ de la rotacion es dos veces
el angulo entre $u$ y $v$.
\\ \\ \\ \\
\item \begin{enumerate}[label=\roman*)]
\item Si deja invariante la recta $r$, entonces $r$ es el eje de la simetria, y tenemos:
\[
r = \begin{pmatrix} 1/2 \\ 0 \end{pmatrix} + \lambda \begin{pmatrix} 1 \\ 1 \end{pmatrix}
\implies
q \in \mathbb{E}^2 = \begin{pmatrix} x \\ y \end{pmatrix} =
\begin{pmatrix} 1/2 \\0 \end{pmatrix} + \lambda \begin{pmatrix} 1 \\1 \end{pmatrix} + 
b \begin{pmatrix} 1 \\ -1 \end{pmatrix}
\implies
b = \frac{x-y-\frac{1}{2}}{2}
\]
(Ya que $\begin{pmatrix} 1 & -1\end{pmatrix} \perp \begin{pmatrix} 1 & 1\end{pmatrix}$). Ahora tenemos
que $S(q) = S(x,y) = \begin{pmatrix} x & y \end{pmatrix} - 2b\begin{pmatrix} 1 & -1 \end{pmatrix}$
(donde $S$ es la simetria, e.d, $f = \tau \circ S$, con $\tau$ una traslacion):
\[
S(x,y) = \begin{pmatrix} x \\ y\end{pmatrix} - 2 \left( \frac{x-y-\frac{1}{2}}{2} \right)
\begin{pmatrix} 1 \\ -1\end{pmatrix} = \begin{pmatrix} y+\frac{1}{2} & x - \frac{1}{2}\end{pmatrix}
\]
Calculamos ahora $S\left(1, -\frac{3}{2}\right) = \begin{pmatrix} -1 & \frac{1}{2}\end{pmatrix}$
Y por lo tanto el vector de traslacion es $\begin{pmatrix} \frac{9}{8} & \frac{9}{8}\end{pmatrix}$, con
lo cual $f(x,y) = \begin{pmatrix} y + \frac{1}{2} & x - \frac{1}{2} \end{pmatrix} +
\begin{pmatrix} \frac{9}{8} & \frac{9}{8} \end{pmatrix} =
\begin{pmatrix} y + \frac{13}{8} & x + \frac{5}{8} \end{pmatrix}$. Entonces $M_e(f)$:
\[
M_e(f) = \begin{pmatrix} 0 & 1 & 13/8 \\ 1 & 0 & 5/8 \\ 0 & 0 & 1\end{pmatrix}
\]
\item Primero encontramos $M_e(Q)$:
\[
M_e(Q) = \begin{pmatrix} 1 & -1 & 0 \\ -1 & 1 & 1 \\ 0 & 1 & 2\end{pmatrix}
\]
Miramos ahora si tiene centros. $p$ es centro sii $Ap +L = 0$:
\[
\begin{pmatrix} 1 & -1 \\ -1 & 1 \end{pmatrix} \begin{pmatrix} x \\ y\end{pmatrix}
+ \begin{pmatrix} 0 \\ 1 \end{pmatrix} = \begin{pmatrix} 0 \\ 0\end{pmatrix}
\]
Que no tiene solucion, por lo tanto $Q$ no tiene centros. Como $\det\tilde{A} \neq 0$, $Q$ es una parabola. Sacamos ahora los $vep$s de $A$:
\[
Q_A(t) = \begin{pmatrix} 1-t & -1 \\ -1 & 1-t\end{pmatrix} = (1-t)^2-1 = t(t-2)
\]
Tomamos entonces $\begin{pmatrix} 1 & 1\end{pmatrix}$ $vep$ de $vap$ 0 y
$\begin{pmatrix} 1 & -1\end{pmatrix}$ $vep$ de $vap$ 2. Normalizamos ahora los $vep$s, quedando
$v_1 = \begin{pmatrix} \frac{1}{\sqrt{2}} & \frac{1}{\sqrt{2}} \end{pmatrix}$ y
$v_2 = \begin{pmatrix} \frac{1}{\sqrt{2}} & -\frac{1}{\sqrt{2}} \end{pmatrix}$. Queremos ahora
$p = \begin{pmatrix} x & y\end{pmatrix}$ tal que $\phi(p,p) = 0$ y $\phi(v_2,p) = 0$, entonces:
\[
\phi(v_2,p) = 0 \iff \begin{pmatrix} \frac{1}{\sqrt{2}} & -\frac{1}{\sqrt{2}} & 0\end{pmatrix}
\begin{pmatrix} 1 & -1 & 0 \\ -1 & 1 & 1 \\ 0 & 1 & 2 \end{pmatrix}\begin{pmatrix} x \\ y \\ 1 \end{pmatrix} =
\begin{pmatrix} 0 \\ 0\end{pmatrix}
\iff
x = y + \frac{1}{2}
\]
Luego $p$ es de la forma $\begin{pmatrix} y + \frac{1}{2} & y \end{pmatrix}$, queremos ahora que:
\[
\left( y + \frac{1}{2} \right)^2 - 2y\left(y+\frac{1}{2} \right) + y^2 + 2y +2 = 0
\iff
y = -\frac{9}{8}
\]
Con lo cual el centro es $p = \begin{pmatrix} -\frac{5}{8} & -\frac{9}{8} \end{pmatrix}$. Tomamos ahora
$\mathcal{R} = \left\{ p, v_1, v_2\right\}$ entonces:
\[
M_{\mathcal{R}}(Q) = \begin{pmatrix} 0 & 0 & b \\ 0 & 2 & 0 \\ b & 0 & 0 \end{pmatrix}
\]
Donde $b = \phi(p,v_1) = \frac{1}{\sqrt{2}}$
\\
\item Calculamos la imagen del vertice, $f(-\frac{5}{8}, -\frac{9}{8}) =
\begin{pmatrix} \frac{1}{2} & 0\end{pmatrix}$. Calculamos ahora la imagen de $p + [v_1]$, que es
$f(p) + [v_1]$ (ya que $p + [v_1]$ es paralelo al eje de simetria). Ahora calculamos $p +[v_2]$, que es
invariante porque $v_2$ es perpendicular al eje de simetria. Por lo tanto, la imagen de $Q$ por $f$, sera
una parabola de vertice $\begin{pmatrix} \frac{1}{2} & 0\end{pmatrix}$ y eje
$\begin{pmatrix} \frac{1}{2} & 0\end{pmatrix} + \left[ \begin{pmatrix} 1 & 1 \end{pmatrix}\right]$.

\end{enumerate}
\end{enumerate}
\end{document}