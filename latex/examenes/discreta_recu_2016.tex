\documentclass{article}

\usepackage[utf8]{inputenc}

\usepackage{geometry}
\geometry{a4paper}

\usepackage{titling}

\usepackage{tikz}
\tikzstyle{every node}=[circle, draw, fill=black!50, inner sep=0pt, minimum width=4pt]

\usepackage{mathtools}
\usepackage{amssymb}
\usepackage{amsfonts}


\begin{document}

\title{Examen final matematica discreta 2016}
\date {}
\maketitle

\begin{enumerate}
\item Sean $m$ y $n$ enteros pares con $1 \leq m \leq n$. Consideramos la rejilla formada por los
puntos $(i, j)$ con $0 \leq i \leq 2n$ y $0 \leq j \leq 2m$. Consideramos caminos en la rejilla
formados por pasos hacia la derecha $(1,0)$ y hacia arriba $(0,1)$,
\begin{enumerate}
\item Dad una formula para el numero de caminos que van de $(0,0)$ a $(2n,2m)$ y que pasan
por el punto $(m,m)$.
\item Dad una formulla para el numero de caminos que van de $(0,0)$ a $(2n,2m)$ y que no pasan
por el punto $A = (n/2,m/2)$ ni por el punto $B = (n, m)$
\end{enumerate}

\item Resolved la recurrencia:
\[
a_{n+1} - \alpha a_n = \beta^n, \quad a_0 = \lambda,
\]
con $\alpha,\beta, \lambda$ numeros reales.

\item Sea $T$ un arbol con $n$ vertices sin vertices de grado 2. Provad que mas de la mitad de los
vertices son hojas.

\item Sea $G$ un grafo con $n$ vertices y $m$ aristas que no contiene triangulos (ciclos de longitud
3)
\begin{enumerate}
\item Si $xy$ es una arista de $G$, provad que $g(x) + g(y) \leq n$.
\item Deducid que \( \sum_{x \in V} g(x)^2 \leq mn \).
\item Utilizad la desigualdad de Cauchy-Schwarz para concluir que $m \leq n^2/4$. \newline
[$\left( \sum_{i=1}^n a_ib_i \right)^2 \leq \left( \sum_{i=1}^n a_i^2 \right)
\left( \sum_{i=1}^n b_i^2 \right), a_i, b_i \in \mathbb{R}.$]
\end{enumerate}
\end{enumerate}

%==========
\newpage
\title{Solucion}
\date{}
\maketitle


\begin{enumerate}
\item

\item Primero resolvemos la recurrencia general:
\[
x - \alpha = 0 \implies x = \alpha.
\]
Si $\alpha \neq \beta$, tenemos:
\[
a_n = A \alpha^n + B\beta^n,
\]
con $a_0 = \lambda$ y $a_1 = 1 + \alpha\lambda$. Y por lo tanto
\[
\begin{cases}
\lambda = A + B \\
1 + \alpha\lambda = A\alpha + B\beta
\end{cases}
\implies
\begin{cases}
A = \lambda - \frac{1}{\beta - \alpha} = \frac{\lambda\beta - \lambda\alpha - 1}{\beta - \alpha} \\
B = \frac{1}{\beta - \alpha}
\end{cases}.
\]
Si $\alpha = \beta$, entonces:
\[
a_n = A\alpha^n + Bn\alpha^n,
\]
con $a_0 = \lambda$ y $a_1 = 1 + \alpha\lambda$.
Por lo tanto:
\[
\begin{cases}
\lambda = A \\
1 + \alpha\lambda = A\alpha + B\alpha \\
\end{cases}
\implies
\begin{cases}
A = \lambda \\
B = \frac{1}{\alpha}
\end{cases}.
\]

\item Sean $\{ x_1, \cdots, x_n \}$ los vertices de $T$, entonces se tiene que:
\[
\sum_{i=1}^n g(x_i) =2n-2
\]
(por ser $T$ un arbol). Sabemos que todo vertice tiene grado al menos 1 (ya que $T$ es conexo)
y suponemos que al menos $n/2$ vertices tienen grado distinto de 1 (que no son hojas), es decir,
grado mayor o igual que 3 (ya que no hay vertices de grado 2). Distinguimos ahora dos casos:

Si $n$ es par, entonces suponemos SPG (Sin Perdida de 
Generalidad) que los vertices $x_1, \cdots, x_{n/2}$ son los que tienen grado mayor o igual que 3.
Entonces:
\[
\sum_{i=1}^{n/2} g(x_i) + \sum_{i=n/2 +1}^n g(x_i) = \sum_{i=1}^n g(x_i) =2n-2.
\]
Pero por otro lado:
\[
2n-2 =\sum_{i=1}^{n/2} g(x_i) + \sum_{i=(n+2)/2}^n g(x_i) \geq 3 \frac{n}{2} + \frac{n}{2} - 1 = 2n-1.
\]
Lo cual es una contradiccion.

Si $n$ es impar, suponemos SPG que los vertices $x_1, \cdots, x_{(n+1)/2}$ son los que tienen
grado mayor o igual que 3. Entonces:
 \[
 \sum_{i=1}^{(n+1)/2} g(x_i) + \sum_{i=(n+3)/2}^n g(x_i) = \sum_{i=1}^n g(x_i) =2n-2.
 \]
 Pero por otro lado:
\[
2n-2 =\sum_{i=1}^{(n+1)/2} g(x_i) + \sum_{i=(n+3)/2}^n g(x_i) \geq
3 \frac{n+1}{2} + \frac{n-1}{2}  = 2n + 1
\]
Lo cual es una contradiccion.

\item \begin{enumerate}
\item Llamamos $S_x$ alconjunto de los vertices adyacentes a $x$ y $S_y$ al conjunto de los
vertices adyacentes a $y$. Entonces, $g(x) + g(y) = \lvert S_x \rvert + \lvert S_y \rvert$, ademas,
$\lvert S_x + S_y \rvert \leq n$. Observamos ahora que $S_x$ y $S_y$ son dijuntos, ya que, si
existiera $z \in S_x$ y $z \in S_y$, entonces el ciclo $xyz$ seria un triangulo. Por lo tanto $S_x$ y
$S_y$ son dijuntos y $g(x) + g(y) = \lvert S_x \rvert + \lvert S_y \rvert = \lvert S_x + S_y \rvert \leq n$

\item Sea $x$ un vertice de $G$, sea $g$ su grado y sean $y_1, \cdots, y_g$ los vertices adyacentes
a $x_0$, entonces:
\[
\left( g(y_1) + g \right) + \cdots + \left( g(y_g) + g \right) \leq gn
\]
(por el apartado a), o, reescribiendo:
\[
g^2 \leq gn - g(y_1) - \cdots - g(y_g),
\]
Llamando $g_x$ a $g(x_1) + \cdots + g(y_g)$, $g(y_0)^2 \leq g(x)n - g_x$. Entonces (siendo 
$z_1, \cdots, z_n$ los vertices de $G$):
\[
\sum_{i=1}^n g(z_i)^2 \leq \sum_{i=1}^n \left( g(z_i)n - g_{z_i} \right) = 2mn - \sum_{i=1}^n g_{z_i}.
\]
Observamos ahora que $\sum_{i=1}^n g_{z_i} = \sum_{i=1}^n g(z_i)^2$, ya que $\forall z_i \in G$,
$g(z_i)$ aparece en la parte izquierda exactamente $g(z_i)$ veces, y por lo tanto son iguales,
sustiyendo en la desigualdad anterior:
\[
\sum_{i=1}^n g(z_i)^2 \leq 2mn - \sum_{i=1}^n g(z_i)^2
\implies
\sum_{i=1}^n g(z_i)^2 \leq mn
\]

\item Sean $z_1, \cdots, z_n$ los vertices de $G$, entonces sea $a_i = g(z_i)$ y $b_i = 1$
($i = 1, \cdots, n$), y se tiene que:
\[
\begin{dcases}
\sum_{i=1}^n a_ib_i =\sum_{i=1}^n g(z_i) = 2m \\
\sum_{i=1}^n a_i^2 = \sum_{i=1}^n g(z_i)^2 \leq mn \\
\sum_{i=1}^n b_i^2 = \sum_{i=1}^n 1 = n
\end{dcases}.
\]
Por lo tanto, aplicanco Cauchy-Schwarz:
\[
4m^2 \leq (mn)(n) = mn^2 \implies m \leq n^2/4
\]
\end{enumerate}
\end{enumerate}
\end{document}
