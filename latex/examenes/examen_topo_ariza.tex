\documentclass[a4paper]{article}

\input{preamble_es}

\begin{document}

\begin{enumerate}[(1)]
    \item
        \begin{enumerate}[a)]
            \item NA
            \item
                Veremos que $f$ es homotopa a constante, por lo tanto, $\mu(f, O) = 0$.
                Como $f$ no es exahustiva, existe $z_0 \in \mathbb{S}^1$ tal que
                $z_0 \notin f\left( \mathbb{S}^1 \right)$, tomamos el punto diametralmente opuesto $z_1$
                y aplicamos el teorema de Poincaré-Bohl, que nos dice que si $O$ no pretenece al segmento que 
                une $f(z)$ y $z_1$, entonces $\mu(f, O) = \mu(z_1, O)$. El único segmento que parte de $z_1$ y
                pasa por $O$, es el que une $z_1$ con $z_0$, como $z_0$ no pertenece a la imagen de $f$, las
                hipótesis del teorema se cumplen y $\mu(f, O) = \mu(z_1, O) = 0$
        \end{enumerate}
    \item
        \begin{enumerate}[a)]
            \item 
                Suponemos que $X \times Y \setminus A \times B$ es no conexo, es decir, $\exists U, V$ abiertos
                tal que $U \cap V = \emptyset$ y $U \cap V = X \times V \setminus A \times B$.
        \end{enumerate}<++>
\end{enumerate}<++>

\end{document}

