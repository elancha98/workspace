\documentclass{article}

\usepackage[utf8]{inputenc}

\usepackage{geometry}
\geometry{a4paper}

\usepackage{titling}

\usepackage{tikz}
\tikzstyle{every node}=[circle, draw, fill=black!50, inner sep=0pt, minimum width=4pt]

\usepackage{mathtools}
\usepackage{amssymb}
\usepackage{amsfonts}

\newcommand{\abs}[1]{\left\lvert #1 \right\rvert}

\begin{document}

\title{Examen extraordinario Calculo I 2013}
\date {}
\maketitle

\begin{enumerate}
\item (3 puntos)
\begin{enumerate}
\item Definid continuidad uniforme. Enunciad y demostrad el teorema de Heine.
\item Probad que la funcion $f(x) = \sin(x/4)$ no es uniformemente continua en $(0,1]$.
\end{enumerate}

\item (3 puntos)
\begin{enumerate}
\item Probad que la funcion $h : (0, \infty) \longrightarrow \mathbb{R}$ dada por 
$h(x) = 2x + 1 - 2\sqrt{x(x+1)}$es decreciente, estrictamente positiva y tiende a 0
cuando $x \to \infty$.
\item Fijado un valor $x > 0$, se considera la ecuacion en $z$:
\[
\sqrt{x+1} - \sqrt{x} = \frac{1}{2\sqrt{x+z}}.
\]
\begin{enumerate}
\item A partor del teorema de Bolzano, probad que hay un $z \in [0,1]$ que es solucion de la
ecuacion anterior
\item Lo mismo, pero aplicando ahora el teorema del Valor Medio
\item Probad tambien que esta solucion es unica
\end{enumerate}
\end{enumerate}
\item (4 puntos) \\
Seaan $g_1, g_2 : [a, b]  \longrightarrow \mathbb{R}$ funciones continuas. Sea tambien
$f : \mathbb{R} \longrightarrow \mathbb{R}$ una funcion continua tal que
$\forall x, 0 < f(x) \leq k < 1$ donde $k$ es una constante. Se define la socesion
$(a_n)_{n \in \mathbb{N}}$ como:
\[
a_n = \int_0^1 \left( f(x) \right)^n dx.
\]
\begin{enumerate}
\item Probad que si $g_1(x) > g_2(x)$, $\forall x \in [a, b]$, entonces
$\displaystyle \int_a^b g_1(x)dx > \int_a^b g_2(x)dx$.
\item Si, ademas, $g_2(x) > 0$, $\forall x \in [a, b]$, probad que existe $c \in [a,b]$, tal que:
\[
\int_a^b g_1(x)g_2(x)dx = g_1(x)\int_a^bg_2(x)dx
\]
\item Probad que $\displaystyle \forall n \in \mathbb{N}, \frac{a_{n+1}}{a_n} \leq k$.
\item Probad que $(a_n)$ es decreciente, esta acotada inferiormente y tiene limite
\item Probad que el limite vale 0
\end{enumerate}
\end{enumerate}

\newpage
%==========================
\title{Solucion}
\date{}
\maketitle

\begin{enumerate}
\item \begin{enumerate}
\item Mirar teoria.
\item Sea $\varepsilon = 0.5$, entonces, $\forall \delta$, existen $x = 1/n, y = 2/(n+1) \in (0, 1]$,
donde $n$ es el menor numero par tal que $3/\delta < (n+1)$. Entonces:
\[
\abs{x - y} = \abs{\frac{1}{n} - \frac{2}{n+1}} = \abs{\frac{3n+1}{n(n+1)}} < \abs{\frac{3n}{n(n+1)}} =
\abs{\frac{3}{n+1}} < \delta.
\]
Pero, por otro lado,
\[
\abs{ \sin{\frac{\pi}{x}} - \sin{\frac{\pi}{y}} } = \abs{\sin n\pi - \sin \frac{(n+1)\pi}{2}} =
\abs{0 - \sin \frac{\pi}{2} } = 1 > 0.5 = \varepsilon.
\]
\\ \\ \\
\end{enumerate}

%===========
\item \begin{enumerate}
\item $h$ es decreciente si y solo si $\forall x, \varepsilon \in (0, \infty)$,
$h(x) - h(x+\varepsilon) \geq 0$,veamos ahora que se cumple:
\[
h(x) - h(x+\varepsilon) = 2\varepsilon - 2\sqrt{x(x+1)} + 2\sqrt{(x+\varepsilon)(x+\varepsilon+1)} =
\]
\[
= 2\varepsilon - 2\sqrt{x^2+x} +
2\sqrt{x^2 + x + 2x\varepsilon+ \varepsilon^2 + \epsilon} \geq 2\varepsilon \geq 0.
\]
Veamos ahora que $h$ es estrictamente positiva:
\[
h(x) = 2x + 1 - 2\sqrt{x(x+1)} = 2\sqrt{x^2 + x + \frac{1}{4}} - 2\sqrt{x^2+x} > 0.
\]
Y que $\displaystyle \lim_{x \to \infty} h(x) = 0$:
\[
\lim_{x \to \infty} h(x) = \lim_{x \to \infty} 2\sqrt{x^2 + x + \frac{1}{4}} - 2\sqrt{x^2+x} = 0
\]
\item \begin{enumerate}
\item Tomamos la funcion $g(z) = 2\sqrt{x^2+x + z(x+1)} - 2\sqrt{x^2+xz} - 1$. Vemos que $g$ es
continua y que
\[
g(0) = 2\sqrt{x^2+x} - 2x - 1 = -h(x) < 0
\]
(como hemos visto en el apartado a), ademas:
\[
g(1) = 2\sqrt{x^2+2x+1} - 2\sqrt{x^2+x} -1 = 2(x+1) - 1 -2\sqrt{x^2+x} = h(x) > 0.
\]
Por lo tanto, por el teorema de Bolzano, existe $z_0 \in [0,1]$, tal que $g(z_0) = 0$, es decir, $z_0$
es solucion de la ecuacion.
\item Primero construimos la funcion $f(z) = \sqrt{x+z}$, que es derivable en $[0,1]$. Ahora, por
el teorema del valor medio, existe $z_0 \in [0,1]$ tal que
$\displaystyle f'(z_0) = \frac{f(1) - f(0)}{1 - 0}$, es decir, existe $z_0$ tal que:
\[
f'(z_0) = \frac{1}{2\sqrt{x+z_0}} = \sqrt{x+1} - \sqrt{x} = f(1) - f(0).
\]
Y por lo tanto, $z_0$ es solucion de la ecuacion.

\item Recuperamos la funcion $g(z) = 2\sqrt{x^2+x + z(x+1)} - 2\sqrt{x^2+xz} - 1$ (definida en el
apartado i), que es derivable en $[0,1]$. Consideramos ahora
\[
g'(z) = \frac{x+1}{\sqrt{x^2+x+z(x+1)}} - \frac{x}{\sqrt{x^2+xz}}.
\]
Y resolvemos la ecuacion $g'(z) = 0$:
\[
g'(z) = 0 \iff \frac{x+1}{\sqrt{x^2+x+z(x+1)}} = \frac{x}{\sqrt{x^2+xz}} \iff
\]
\[
\iff
(x+1)\sqrt{x^2+xz} = x\sqrt{x^2+x+z(x+1)} \iff
\]
\[
\iff
x^4 + zx^3 + 2x^3 + 2zx^2 + x^2 +xz = x^4 +zx^3+x^3+zx^2
\iff
x^3 + zx^2 + x^2 +xz = 0
\]
Que no tiene solucion ya que $x,z>0$. Por lo tanto la ecuacion $g'(z) = 0$ no tiene ninguna
solcuion en el intervalo $[0,1]$  y por consiguiente, $g(z)$ tiene como mucho una raiz (en ese
intervalo), la ya encontrada $z_0$.
\end{enumerate}
\end{enumerate}
\end{enumerate}
\end{document}
