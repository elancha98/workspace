\documentclass[a4paper]{article}

\usepackage[utf8]{inputenc}
\usepackage[T1]{fontenc}
\usepackage[a4paper,margin=1in]{geometry}
\usepackage[pdftex]{hyperref}
\usepackage{amsmath,amsthm,amssymb,graphicx,mathtools,tikz,hyperref,enumerate}
\usepackage{mdframed,cleveref,cancel,stackengine,pgf,pgfplots,mathrsfs,thmtools}
\usepackage{xfrac,stmaryrd,commath,needspace,multirow,float}
\usepackage[shortlabels]{enumitem}
\usepackage[spanish,es-lcroman]{babel}

\newmdenv[leftline=false,topline=false]{topright}
\let\proof\relax
\usetikzlibrary{positioning,arrows, calc, babel}
\usetikzlibrary{external}
\tikzexternalize[prefix=figures/]
\pgfplotsset{compat=1.11}

\newcommand*{\bimplies}{\boxed{\implies}}
\newcommand*{\bimpliedby}{\boxed{\impliedby}}

\newcommand{\n}{\mathbb{N}}
\newcommand{\z}{\mathbb{Z}}
\newcommand{\q}{\mathbb{Q}}
\newcommand{\cx}{\mathbb{C}}
\newcommand{\real}{\mathbb{R}}
\newcommand{\E}{\mathbb{E}}
\newcommand{\F}{\mathbb{F}}
\newcommand{\A}{\mathbb{A}}
\newcommand{\R}{\mathcal{R}}
\newcommand{\C}{\mathscr{C}}
\newcommand{\Pa}{\mathcal{P}}
\newcommand{\Es}{\mathcal{E}}
\newcommand{\V}{\mathcal{V}}
\newcommand{\T}{\mathcal{T}}
\newcommand{\B}{\mathcal{B}}
\newcommand{\bb}[1]{\mathbb{#1}}
\newcommand{\pdv}[3][]{\frac{\partial^{#1} #2}{\partial #3^{#1}}}
\newcommand{\dv}[3][]{\frac{\dif^{#1} #2}{\dif #3^{#1}}}
\let\k\relax
\newcommand{\k}{\Bbbk}
\newcommand{\ita}[1]{\textit{#1}}
\newcommand\inv[1]{#1^{-1}}
\newcommand\setb[1]{\left\{#1\right\}}
\newcommand{\vbrack}[1]{\langle #1\rangle}
\newcommand{\determinant}[1]{\begin{vmatrix}#1\end{vmatrix}}
\newcommand{\Po}{\mathbb{P}}
\newcommand{\lp}{\left(}
\newcommand{\rp}{\right)}
\newcommand{\lc}{\left\{}
\newcommand{\rc}{\right\}}
\DeclareMathOperator{\fr}{Fr}
\DeclareMathOperator{\Id}{Id}
\DeclareMathOperator{\ext}{Ext}
\DeclareMathOperator{\inte}{Int}
\DeclareMathOperator{\rie}{Rie}
\DeclareMathOperator{\rg}{rg}
\DeclareMathOperator{\gr}{gr}
\DeclareMathOperator{\nuc}{Nuc}
\DeclareMathOperator{\car}{car}
\DeclareMathOperator{\im}{Im}
\DeclareMathOperator{\tr}{tr}
\DeclareMathOperator{\vol}{vol}
\DeclareMathOperator{\grad}{grad}
\DeclareMathOperator{\rot}{rot}
\DeclareMathOperator{\diver}{div}
\DeclareMathOperator{\sinc}{sinc}
\DeclareMathOperator{\graf}{graf}
\DeclareMathOperator{\tq}{\;t.q.\;}
\DeclareMathOperator{\disc}{disc}
\let\emptyset\varnothing
\setcounter{secnumdepth}{4}

\def\mydate{\today}
\hypersetup{
    colorlinks,
    linkcolor=blue
}
\def\upint{\mathchoice%
    {\mkern13mu\overline{\vphantom{\intop}\mkern7mu}\mkern-20mu}%
    {\mkern7mu\overline{\vphantom{\intop}\mkern7mu}\mkern-14mu}%
    {\mkern7mu\overline{\vphantom{\intop}\mkern7mu}\mkern-14mu}%
    {\mkern7mu\overline{\vphantom{\intop}\mkern7mu}\mkern-14mu}%
  \int}
\def\lowint{\mkern3mu\underline{\vphantom{\intop}\mkern7mu}\mkern-10mu\int}

\newtheoremstyle{break}% name
{}%         Space above, empty = `usual value'
{}%         Space below
{}% Body font
{}%         Indent amount (empty = no indent, \parindent = para indent)
{\bfseries}% Thm head font
{}%        Punctuation after thm head
{\newline}% Space after thm head: \newline = linebreak
{\thmname{#1}\thmnumber{ #2}.\thmnote{ {\it #3.}}}%         Thm head spec

\newtheoremstyle{demo}% name
{}%         Space above, empty = `usual value'
{}%         Space below
{}% Body font
{}%         Indent amount (empty = no indent, \parindent = para indent)
{\it}% Thm head font
{}%        Punctuation after thm head
{5pt plus 1pt minus 1pt}% Space after thm head: \newline = linebreak
{#1\thmnote{ #3}.}%         Thm head spec

\newtheoremstyle{breakthm}% name
{}%         Space above, empty = `usual value'
{}%         Space below
{}% Body font
{}%         Indent amount (empty = no indent, \parindent = para indent)
{\bfseries}% Thm head font
{}%        Punctuation after thm head
{\newline}% Space after thm head: \newline = linebreak
{#1 \normalfont #3 (#2)\addcontentsline{toc}{subsection}{#1 #3}}%         Thm head spec

\newtheoremstyle{normal}% name
{}%         Space above, empty = `usual value'
{}%         Space below
{}% Body font
{}%         Indent amount (empty = no indent, \parindent = para indent)
{\bfseries}% Thm head font
{}%        Punctuation after thm head
{5pt plus 1pt minus 1pt}% Space after thm head: \newline = linebreak
{\thmname{#1}\thmnumber{ #2}.\thmnote{ {\it #3.}}}%         Thm head spec

% Normal
\declaretheorem[style=normal,name=Lema,numberwithin=section]{lema}
\declaretheorem[style=normal,name=Lema,numbered=no]{lema*}
\declaretheorem[style=normal,name=Observación,sibling=lema]{obs}
\declaretheorem[style=normal,name=Observación,numbered=no]{obs*}
\declaretheorem[style=normal,name=Proposición,sibling=lema]{prop}
\declaretheorem[style=normal,name=Proposición,numbered=no]{prop*}
\declaretheorem[style=normal,name=Definición,sibling=lema]{defi}
\declaretheorem[style=normal,name=Definición,numbered=no]{defi*}
\declaretheorem[style=normal,name=Corolario,sibling=lema]{col}
\declaretheorem[style=normal,name=Corolario,numbered=no]{col*}
\declaretheorem[style=normal,name=Ejercicio,sibling=lema]{ej}
\declaretheorem[style=normal,name=Ejercicio,numbered=no]{ej*}
\declaretheorem[style=normal,name=Ejemplo]{example}
\declaretheorem[style=normal,name=Ejemplo,numbered=no]{example*}

% Demo
\declaretheorem[style=demo,name=Demostración,qed=$\square$,numbered=no]{proof}

% Break
\declaretheorem[style=break,name=Teorema,sibling=lema]{teo*}

% Breakthm
\declaretheorem[style=breakthm,name=Teorema,sibling=lema]{teo}
\declaretheorem[style=breakthm,name=Lema,numberwithin=section]{teolema}

\begin{document}

\title{Resolución de EDOs lineales simples}
\date{}
\maketitle

Lo primero es que este método tan solo nos servirá para resolver EDOs 
simples y lineales, en particular, nos será muy útil para la asignatura
de física de segundo.

Primero, tenemos que tener planteada la edos, llamaremos $f$ a la función
que queremos encontrar (aunque en física es común que sea $v$ o $r$) y el resto
de variables serán constantes. Nos debe quedar una ecuación de este tipo:

\[
  a_n \dv[n]{f(t)}{t} +
  a_{n-1} \dv[n-1]{f(t)}{t} +
  \cdots + a_2 \dv[2]{f(t)}{t} + a_1 \dv{f(t)}{t}
  + a_0 f(t) = b
\]

\textbf{Nota:} Si $a_0, a_1, \dots, a_m$ son 0, pasamos $b$ al otro lado e integramos
hasta obtener algo de la forma

\[
  \begin{gathered}
    a_{n} \dv[n]{f(t)}{t} +
    a_{n-1} \dv[n-1]{f(t)}{t} +
    \cdots + a_m \dv[m]{f(t)}{t} = b
    \\
    a_{n} \dv[n-m]{f(t)}{t} +
    a_{n-1} \dv[n-m-1]{f(t)}{t} +
    \cdots + a_m f(t) = b t^m
  \end{gathered}
\]

Ahora, encontramos la solución de la EDO homogénea:
\[
  a_n \dv[n]{f(t)}{t} + \cdots + a_1 \dv{f(t)}{t} + a_0 f(t) = 0
\]
Para ello, utilizamos el polinomio característico:
\[
  \begin{aligned}
    & a_n \dv[n]{f(t)}{t} & + & a_{n-1} \dv[n-1]{f(t)}{t} & +
    \cdots & + a_0 f(t) & = 0 \\
    & a_n x^n & + & a_{n-1} x^{n-1} & + \cdots & + a_0 & = 0
  \end{aligned}
\]
Ahora, resolvemos el polinomio. Aquí asumiremos que todas las raices son reales
y de multiplicidad 1 (para otros casos consultar el libro de física, final de la paǵina 16).
Nos quedan $n$ raices $r_1, r_2, \dots, r_n$. Y la solución de la ecuación homogénea es:
\[
  f_{\text{hom}}(t) = c_1 e^{r_1 t} + c_2 e^{r_2 t} + \cdots + c_n e^{r_n tr_n t}
\]

Lo siguiente es encontrar una solución particular, que será un polinomio del mismo grado 
que el término independiente, de tal forma que si nuestra EDOs era de la forma
\[
  a_{n} \dv[n-1]{f(t)}{t} +
  a_{n-1} \dv[n-1]{f(t)}{t} +
  \cdots + a_0 f(t) = b t^m
\]

Probaremos una solución de la forma
\[
  f_{\text{part}} = d_m t^m + d_{m-1} t^{m-1} + \cdots + d_1 t + d_0 
\]

Y ahora, tenemos que ajustar los parámetros $d_0, \dots, d_{m-1}$,
substituyendo en la EDO:
\[
  a_m d_m + a_{m-1} \left( d_m t + d_{m-1} \right) + \cdots +
  a_{0} \left( d_m t^m + d_{m-1} t^{m-1} + \cdots + d_1 t + d_0 \right) = b t^m
\]

Ahora, igualando coeficiente a coeficiente, obtendremos un sistema lineal de donde
obtendremos $d_0, \dots, d_m$.

Una vez realizados estos dos pasos, ya estamos preparados para encontrar nuestra $f$:
\[
  f = f_{\text{hom}} + f_{\text{part}}
\]

Por último, nos falta ajustar los coeficientes $c_1, \dots, c_n$, para ello, damos $n$
valores a f y substituimos (el resultado de estos valores debe ser conocido). Es decir,
para poder dar una solución única para la EDO, tenemos que conocer
\[
  f(t_1) = x_1 \quad f(t_2) = x_2 \quad \cdots \quad f(t_n) = x_n
\]

Resolver los valores de $c_1, \dots, c_n$ es fácil con está información, tan solo
substituimos y resolvemos el sistema lineal.

\section{Ejemplos}

\begin{example}
  Encontrar $v$
  \[
    m \dv{v}{t} = C - bv
  \]
  Sabiendo que $v(0) = 0$.
\end{example}
\begin{proof}
  Primero nos ponemos la EDO de la forma
  \[
    m \dv{v}{t} + bv = C
  \]
  Y resolvemos la forma homogénea, con el polinomio
  \[
    m \dv{v}{t} + bv = 0 \qquad m x + b = 0
  \]
  La raiz del polinomio es $x = \frac{-b}{m}$, por lo tanto
  \[
    v_{\text{hom}} = c_1 e^{\frac{-b}{m} t}
  \]

  Lo siguiente es encontrar una solución particular (de grado 0), tomamos
  \[
    v_{\text{part}} = d_0
  \]

  Y nos queda
  \[
    m 0 + b d_0 = C \implies d_0 = \frac{C}{b}
  \]
  Por lo tanto tenemos que
  \[
    v = v_{\text{part}} + v_{\text{hom}} = \frac{C}{b} + c_1 e^{\frac{-b}{m} t}
  \]

  Ahora solo queda encontrar $c_1$, para ello tenemos
  \[
    v(0) = \frac{C}{b} + c_1 e^{\frac{-b}{m} 0} = 0 \implies c_1 = -\frac{C}{b}
  \]
  Con lo que concuimos que 
  \[
    v(t) = \frac{C}{b} - \frac{C}{b} e^{\frac{-b}{m} t}
  \]
\end{proof}

\begin{example}
  Encontrar $r$, sabiendo que
  \[
    m^2 \dv[2]{r}{t} = C + r
  \]
  Y que $r(0) = 0$ y $r(1) = -C$.
\end{example}
\begin{proof}
  Primero, ponemos la EDO como
  \[
    m^2 \dv[2]{r}{t} - r = C
  \]
  Ahora tomamos la homogénea y calculamos el polinomio:
  \[
    m^2 \dv[2]{r}{t} - r = 0 \qquad m^2 x^2 - 1 = 0
  \]

  Las raices del polinomio son $x = \frac{\pm 1}{m}$, por lo tanto
  \[
    r_{\text{hom}} = c_1 e^{\frac{1}{m} t} + c_2 e^{\frac{- 1}{m} t}
  \]

  Ahora calculamos $r_{\text{part}}$ que es un polinomio de grado 0:
  \[
    r_{\text{part}} = d_0
  \]

  Al substituir:
  \[
    m^2 0 - d_0 = C \implies d_0 = -C
  \]

  Ahora solo queda sumar:
  \[
    v = v_{\text{hom}} + v_{\text{part}} = c_1 e^{\frac{1}{m} t} + c_2 e^{\frac{-1}{m} t} - C
  \]

  Y ajustar $c_1$ y $c_2$:
  \[
    \begin{cases}
      v(0) = 0 \implies c_1 + c_2 = C \\
      v(1) = -C \implies e c_1 + \frac{1}{e} c_2 = 0
    \end{cases}
    \implies
    \begin{cases}
      c_1 = \frac{C}{1 - e^2} \\
      c_2 = \frac{-Ce^2}{1-e^2}
    \end{cases}
  \]

  Por lo tanto,
  \[
    v(t) = \frac{C}{1 - e^2} e^{\frac{1}{m} t} + \frac{-Ce^2}{1-e^2} e^{\frac{-1}{m} t} - C
  \]
\end{proof}

\begin{example}
  Encontrar $v$ sabiendo que
  \[
    m \dv{v}{t} = C t - v
  \]
  Y $v(0) = 1 - Cm$
\end{example}
\begin{proof}
  Tomamos la forma siguiente
  \[
    m \dv{v}{t} + v = C t
  \]

  Resolvemos la homogenea encontrando el polinomio
  \[
    m \dv{v}{t} + v = 0 \qquad m x + 1 = 0
  \]

  La raiz es $x = \frac{1}{m}$, y por lo tanto:
  \[
    v_{\text{hom}} = c_1 e^{\frac{1}{m} t}
  \]

  La solucion particular es un polinomio de grado 1:
  \[
    v_{\text{part}} = d_1 t + d_0
  \]

  Encontramos ahora $d_0$ y $d_1$:
  \[
    \begin{gathered}
      m \dv{\left( d_1 t + d_0 \right)}{t} + \left( d_1 t + d_0  \right) =
      m d_1 + d_1 t + d_0 = C t
      \implies \\
      \begin{cases}
	d_1 = C \\
	m d_1 + d_0 = 0
      \end{cases}
      \implies
      \begin{cases}
	d_1 = C \\
	d_0 = -Cm
      \end{cases}
    \end{gathered}
  \]

  Ahora
  \[
    v = v_{\text{hom}} + v_{\text{part}} = c_1 e^{\frac{t}{m}} + C t - Cm
  \]

  Solo queda ajustar $c_1$:
  \[
    v(0) = 1 - Cm \implies c_1 - Cm = 1 - Cm \implies c_1 = 1
  \]

  con lo cual
  \[
    v = e^{\frac{t}{m}} + Ct - Cm
  \]
\end{proof}


\end{document}
