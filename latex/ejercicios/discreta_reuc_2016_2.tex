\documentclass{article}

\usepackage[utf8]{inputenc}

\usepackage{geometry}
\geometry{a4paper}

\usepackage{titling}

\usepackage{tikz}
\tikzstyle{every node}=[circle, draw, fill=black!50, inner sep=0pt, minimum width=4pt]

\usepackage{amsmath}
\usepackage{amssymb}
\usepackage{amsfonts}


\begin{document}
\noindent
2.) Resolved la recurrencia:
\[
a_{n+1} - \alpha a_n = \beta^n, \quad a_0 = \lambda,
\]
con $\alpha,\beta, \lambda$ numeros reales.
\\ \\ \\ \\ \\ \\
Primero resolvemos la recurrencia general:
\[
x - \alpha = 0 \implies x = \alpha.
\]
Si $\alpha \neq \beta$, tenemos:
\[
a_n = A \alpha^n + B\beta^n,
\]
con $a_0 = \lambda$ y $a_1 = 1 + \alpha\lambda$. Y por lo tanto
\[
\begin{cases}
\lambda = A + B \\
1 + \alpha\lambda = A\alpha + B\beta
\end{cases}
\implies
\begin{cases}
A = \lambda - \frac{1}{\beta - \alpha} = \frac{\lambda\beta - \lambda\alpha - 1}{\beta - \alpha} \\
B = \frac{1}{\beta - \alpha}
\end{cases}.
\]
Si $\alpha = \beta$, entonces:
\[
a_n = A\alpha^n + Bn\alpha^n,
\]
con $a_0 = \lambda$ y $a_1 = 1 + \alpha\lambda$.
Por lo tanto:
\[
\begin{cases}
\lambda = A \\
1 + \alpha\lambda = A\alpha + B\alpha \\
\end{cases}
\implies
\begin{cases}
A = \lambda \\
B = \frac{1}{\alpha}
\end{cases}.
\]
\end{document}
