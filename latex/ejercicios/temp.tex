\documentclass[a4paper]{article}

\input{preamble_es}

\begin{document}

Dada la serie
\[
    \sum \frac{2^n \sin(\alpha)^{2n}}{n^\beta}
\]
Estudiaremos su convergencia.

Primero, miraremos la convergencia absoluta:
\[
    \sum \abs{\frac{2^n \sin(\alpha)^{2n}}{n^\beta}} =
    \sum \abs{\frac{2 \sin(\alpha)^2}{n^{\frac{\beta}{n}}}}^n 
\]
Además, como todos los números son positivos, podemos quitar el valor absoluto y 
se tiene la siguiente desigualdad
\[
    \sum \left( \frac{2 \sin(\alpha)^2}{n^{\frac{\beta}{n}}} \right)^n \leq
    \sum \left( 2 \sin(\alpha)^2 \right)^n
\]
Ahora, si $2 \sin(\alpha)^2 < 1$, la serie, claramente converge. Por otro lado, si
$2 \sin(\alpha)^2 > 1$, $\exists N \in \n$ tal que
\[
    \forall n > N \qquad 2 \sin(\alpha)^2 > n^{\frac{\beta}{n}} \implies
    \frac{2 \sin(\alpha)^2}{n^{\frac{\beta}{n}}} > 1
\]
Y por lo tanto, la serie diverge. Por último, si $2 \sin(\alpha)^2 = 1$, tenemos
\[
    \sum \frac{1}{n^\beta}
\]
Y por lo tanto la serie converge sii $\beta > 1$.

\newpage

\begin{enumerate}[a)]
    \item Primero supodremos que la conición del enunciado es falsa, es decir, que no existe $W$. Entonces
        tenemos que $\forall W \ni y$, $f^{-1}(W) \not\subseteq U \implies \exists x \in \inv{f}(W) \setminus U$
        tal que $f(x) \in W$.
\end{enumerate}<++>

\end{document}

