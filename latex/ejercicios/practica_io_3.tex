\documentclass[12pt,a4paper,table]{article}

\usepackage[margin=1in]{geometry}
\usepackage[pdftex]{hyperref}
\usepackage[table]{xcolor}
\usepackage{amsmath,amsthm,amssymb,graphicx,mathtools,tikz,hyperref,enumerate}
\usepackage{mdframed,cleveref,cancel,stackengine,pgfplots,subcaption,pgf,xfrac}
\usepackage{listings}
%\usepackage[spanish]{babel}

\newmdenv[leftline=false,topline=false]{topright}
\let\proof\relax
\usepackage[utf8]{inputenc}
\usetikzlibrary{positioning}
\newcommand{\n}{\mathbb{N}}
\newcommand{\z}{\mathbb{Z}}
\newcommand{\q}{\mathbb{Q}}
\newcommand{\cx}{\mathbb{C}}
\newcommand{\real}{\mathbb{R}}
\newcommand{\E}{\mathbb{E}}
\newcommand{\F}{\mathbb{F}}
\newcommand{\D}{\mathcal{D}}
\newcommand{\bb}[1]{\mathbb{#1}}
\let\k\relax
\newcommand{\k}{\mathbf{k}}
\newcommand{\ita}[1]{\textit{#1}}
\newcommand{\inv}[1]{#1^{-1}}
\newcommand\setb[1]{\left\{#1\right\}}
\newcommand{\vbrack}[1]{\langle #1\rangle}
\newcommand{\determinant}[1]{\begin{vmatrix}#1\end{vmatrix}}
\newcommand{\abs}[1]{\left\vert #1 \right\vert}
\newcommand{\Po}{\mathbb{P}}

\newcommand{\dualpar}[1]{\left(#1\right)^\ast}
\newcommand{\dual}[1]{#1^\ast}

\DeclareMathOperator{\join}{\vee}
\DeclareMathOperator{\Id}{Id}
\DeclareMathOperator{\rg}{rg}
\DeclareMathOperator{\car}{car}

\newcommand{\myref}[1]{\hyperref[item:a#1]{\uppercase{a#1}}}


\hypersetup{
	colorlinks,
	linkcolor=blue
}

\definecolor{codegreen}{rgb}{0,0.6,0}
\definecolor{codegray}{rgb}{0.5,0.5,0.5}
\definecolor{codepurple}{rgb}{0.58,0,0.82}
\definecolor{backcolour}{rgb}{0.95,0.95,0.92}
\definecolor{purple}{HTML}{7F0055}

\lstdefinestyle{mystyle}{
    backgroundcolor=\color{backcolour},
    commentstyle=\color{codegreen},
    keywordstyle=\color{magenta},
    numberstyle=\tiny\color{codegray},
    stringstyle=\color{codepurple},
    escapechar=|,
    basicstyle=\footnotesize,
    breakatwhitespace=false,
    breaklines=true,
    captionpos=b,
    keepspaces=true,
    numbers=left,
    numbersep=5pt,
    showspaces=false,
    showstringspaces=false,
    showtabs=false,
    tabsize=2
}

\lstset{style=mystyle}

\usetikzlibrary{arrows}

\newtheoremstyle{break}% name
{}%         Space above, empty = `usual value'
{}%         Space below
{}% Body font
{}%         Indent amount (empty = no indent, \parindent = para indent)
{\bfseries}% Thm head font
{}%        Punctuation after thm head
{\newline}% Space after thm head: \newline = linebreak
{#1 #2 \normalfont \textit{#3}}%         Thm head spec


\theoremstyle{break}
\newtheorem{ej}{Ejercicio}
\newtheorem{defi}{Definición}
\let\proof\relax
\newtheorem*{proof}{Solución}

%\pgfplotsset{compat=1.5}

\begin{document}
\date{\today}

\title{\textbf{Programación lineal entera} \\
\small{Calendario de la National Football League}}
\author{Marc Esquerrà \and Ernesto Lanchares}
\maketitle

\tableofcontents
\newpage
\section{Formulación matemática}

Primero, vamos a definir nuestras variables. Tendremos una variable por cada posible
partido por cada jornada, es decir, nuestro conjunto de variables son
\[
	\texttt{VAR} = \setb{x_{ijk} \left\vert \substack{1 \leq i \leq n \\
		1 \leq j \leq n \\ i < j \\ 1 \leq k \leq r(n/2- 1) +
		sn/2}\right.}
\]
Donde cada $x_{ijk}$ es una variable binaria, que vale 1 si, en la jornada $k$ el
equipo $i$ se enfrenta al equipo $j$ y 0 en otro caso.

De aquí, podemos deducir nuestro vector de costos, asignando un $c_{ijk}$ a cada
$x_{ijk}$ creando así nuestra función objetivo.
\[
	c_{ijk} =
	\begin{cases}
		0 \text{ si $(i,j)$ es un partido inter-divisional.} \\
		0 \text{ si $(i,j)$ es un partido intra-dimensional y $k=1$} \\
		2^{k-2} \text{ si $(i,j)$ es un partido intra-dimensional y $k \geq 2$}
	\end{cases}
\]
(Los valores de $c_{ijk}$ coinciden con los del enunciado debido a la forma en la que
hemos definido nuestro conjunto de variables).

Por tanto, tan solo nos queda definir nuestras restricciones. Primero, impondremos
que cada equipo juege un partido cada semana, para ello, necesitamos que
\[
	\sum^n_{j = i+1} x_{ijk} +
	\sum^{i-1}_{j = 1} x_{jik} = 1 \qquad
	\substack{\forall k=1,\dots,r(n/2-1)+sn/2 \\ \forall i = 1,\dots,n}
\]
Asegurandonos que cada equipo juega exactamente un partido por jornada. Ahora, hemos de
asegurar que cada equipo $i$ juega exactamente los partidos que le tocan contra cada
equipo, así pues, tenemos
\[
	\sum^{r(n/2 - 1) + sn/2}_{k=1} x_{ijk} =
	\begin{cases}
		r \text{ si $i$ y $j$ están en la misma división} \\
		s \text{ si $i$ y $j$ están en divisiones distintas}
	\end{cases}
\]
Impondremos está restricción para cada parejas de equipos, es decir, $\forall i =
1,\dots,n$, $\forall j = 1, \dots, (i-1)$.

De forma que, el problema final queda así
\[
	\begin{cases}\displaystyle
		\textbf{max } \sum^{jor}_{k=1} \sum^n_{i=1}
		\sum^{n}_{j=i+1} c_{ijk} x_{ijk} \\
		\textbf{s.a.} \\
		\displaystyle
		\indent\sum^n_{j = i+1} x_{ijk} + \sum^{i-1}_{j = 1} x_{jik} = 1 \qquad
		\substack{\forall k=1,\dots,jor \\ \forall i = 1,\dots,n} \\
		\displaystyle
		\indent\sum^{jor}_{k=1} x_{ijk} = r \qquad
		\substack{\forall i=1,\dots,n/2 \\ \forall j=i+1,\dots,n/2} \\
		\displaystyle
		\indent\sum^{jor}_{k=1} x_{ijk} = r \qquad
		\substack{\forall i=(n/2)+1,\dots,n \\ \forall j=i+1,\dots,n} \\
		\displaystyle
		\indent\sum^{jor}_{k=1} x_{ijk} = s \qquad
		\substack{\forall i=1,\dots,n/2 \\ \forall j=(n/2)+1,\dots,n}
	\end{cases}
\]
Identificando $jor = r(n/2 - 1) + sn/2$.

\section{Modelo en AMPL}

\begin{lstlisting}[language=Awk]
|\color{purple}\textbf{param}| n;
|\color{purple}\textbf{param}| r;
|\color{purple}\textbf{param}| s;
|\color{purple}\textbf{param}| jornada:= r*(n/2 -1) + n*s/2;

|\color{purple}\textbf{set}| Div1:=1..n/2;
|\color{purple}\textbf{set}| Div2:=(n/2+1)..n;
|\color{purple}\textbf{set}| jornadas:=1..jornada;
|\color{purple}\textbf{set}| equipo = 1..n;
|\color{purple}\textbf{set}| partidoIntra1:= {i |\color{purple}\textbf{in}| Div1, j |\color{purple}\textbf{in}| Div1: i < j};
|\color{purple}\textbf{set}| partidoIntra2:= {i |\color{purple}\textbf{in}| Div2, j |\color{purple}\textbf{in}| Div2: i < j};
|\color{purple}\textbf{set}| partidoInter := {i |\color{purple}\textbf{in}| Div1, j |\color{purple}\textbf{in}| Div2};
|\color{purple}\textbf{set}| partidos := partidoIntra1 |\color{purple}\textbf{union}| partidoIntra2 |\color{purple}\textbf{union}| partidoInter;

|\color{purple}\textbf{var}| x {(i,j) |\color{purple}\textbf{in}| partidos, k |\color{purple}\textbf{in}| jornadas}>=0,<=1,|\color{purple}\textbf{integer};|

|\color{purple}\textbf{maximize}| total:
	|\color{purple}\textbf{sum}|{k |\color{purple}\textbf{in}| 2..jornada}(|\color{purple}\textbf{sum}|{(i,j) |\color{purple}\textbf{in}| partidoIntra1} (2^(k-2))*x[i,j,k] + |\color{purple}\textbf{sum}|{(i,j) |\color{purple}\textbf{in}| partidoIntra1} (2^(k-2)*x[i,j,k]));
	
|\color{purple}\textbf{subject to}| res_jornadas{k |\color{purple}\textbf{in}| jornadas, a |\color{purple}\textbf{in}| equipo}:
	(|\color{purple}\textbf{sum}|{(a,j) |\color{purple}\textbf{in}| partidos} x[a,j,k]) + (sum{(i,a) |\color{purple}\textbf{in}| partidos} x[i,a,k]) = 1;

|\color{purple}\textbf{subject to}| res_equipos1{(i,j) |\color{purple}\textbf{in}| partidoIntra1}:
	(|\color{purple}\textbf{sum}|{k |\color{purple}\textbf{in}| jornadas} x[i,j,k])=r;
	
|\color{purple}\textbf{subject to}| res_equipos2{(i,j) |\color{purple}\textbf{in}| partidoIntra2}:
	(|\color{purple}\textbf{sum}|{k |\color{purple}\textbf{in}| jornadas} x[i,j,k])=r;
	
|\color{purple}\textbf{subject to}| res_equipos{(i,j) |\color{purple}\textbf{in}| partidoInter}:
	(|\color{purple}\textbf{sum}|{k |\color{purple}\textbf{in}| jornadas} x[i,j,k])=s;
\end{lstlisting}

\section{Ejemplos}
\subsection{$n = 6, r =  2, s = 3$}
El resultado de la función objetivo en este caso es 8064, quedando las jornadas de
esta  manera:

\begin{center}\begin{tabular}[]{|rcl|rcl|rcl|rcl|}
	\hline
	\rowcolor{gray!50} Jornada 1 & \hspace{1em} & \hspace{1em} &
	Jornada 2 & \hspace{1em} & \hspace{1em} &
	Jornada 3 & \hspace{1em} & \hspace{1em} &
	Jornada 4 & \hspace{1em} & \hspace{1em} \\ \hline
	1 & - & 5  &  1 & - & 4  &  1 & - & 6  &  1 & - & 4 \\ \hline
	2 & - & 6  &  2 & - & 5  &  2 & - & 5  &  2 & - & 6 \\ \hline
	3 & - & 4  &  3 & - & 6  &  3 & - & 4  &  3 & - & 5 \\ \hline
	\rowcolor{gray!50} Jornada 5 & \hspace{1em} & \hspace{1em} &
	Jornada 6 & \hspace{1em} & \hspace{1em} &
	Jornada 7 & \hspace{1em} & \hspace{1em} &
	Jornada 8 & \hspace{1em} & \hspace{1em} \\ \hline
	1 & - & 4  &  1 & - & 6  &  1 & - & 6  &  1 & - & 3 \\ \hline
	2 & - & 6  &  2 & - & 5  &  2 & - & 4  &  2 & - & 4 \\ \hline
	3 & - & 5  &  3 & - & 4  &  3 & - & 5  &  5 & - & 6\\ \hline
	\rowcolor{gray!50} Jornada 9 & \hspace{1em} & \hspace{1em} &
	Jornada 10 & \hspace{1em} & \hspace{1em} &
	Jornada 11 & \hspace{1em} & \hspace{1em} &
	Jornada 12 & \hspace{1em} & \hspace{1em} \\ \hline
	1 & - & 3  &  1 & - & 2  &  1 & - & 2  &  1 & - & 5 \\ \hline
	2 & - & 4  &  3 & - & 6  &  3 & - & 6  &  2 & - & 3 \\ \hline
	5 & - & 6  &  4 & - & 5  &  4 & - & 5  &  4 & - & 6 \\ \hline
	%\rowcolor{gray!50} Jornada 13 & \hspace{1em} & \hspace{1em} \\ \hline
	%1 & - & 5 \\ \hline
	%2 & - & 3 \\ \hline
	%4 & - & 6 \\ \hline
\end{tabular}\end{center}


\end{document}
