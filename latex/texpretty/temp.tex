\begin{eje}
    Tenemos que comprobar que 
    \begin{enumerate}[i)]
        \item $d(a,b) \geq 0 \; \forall a,b 
\in A^n$ y
        que $d(a,b) = 0 \iff a = b$, lo 
cual es trivial
        por la definición de $d$. \item 
$d(a,b) = d(b,a)
        \; \forall a,b \in A^n$. Trivial 
por la 
        definición. \item $d(a, b) \leq 
d(a,c) + d(c, 
        b)$. Sea $i \in R = \setb{j \vert 
a_j \neq 
        b_j}$, entonces, se cumple al menos 
una de las
        dos siguientes afirmaciones 
        \[
            
            \begin{cases}
                i \in P = 
                \setb{j \vert 
                a_j \neq c_j} \\
                i \in Q = 
                \setb{j \vert 
                c_j \neq b_j} 
            \end{cases}
            
        \]
        Por lo tanto, $R \subseteq P \cup Q 
\implies 
        d(a,b) = \abs{R} \leq \abs{P \cup 
Q} \leq 
        \abs{P} + \abs{Q} = d(a,c) + 
d(a,b)$. 
    \end{enumerate}
    
\end{eje}
