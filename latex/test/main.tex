\documentclass{article}
\usepackage[utf8]{inputenc}

\usepackage{amsmath}
\usepackage{amsfonts}

\newcommand\norm[1]{\left\lVert#1\right\rVert}
\newcommand\pdv[2]{\frac{\partial#1}{\partial#2}}

\begin{document}
\[
Jf(x, y, z) = 
\begin{pmatrix}
2xy + e^x & x^2 & 1
\end{pmatrix}
\implies
Jf(0,1,-1) =
\begin{pmatrix}
1 & 0 & 1
\end{pmatrix}
\]
Observamos que $det(\frac{\partial f}{\partial x}) \neq o$ y que $f$ es de clase $C^\infty$, por lo tanto:
\[
\begin{aligned}
	\exists  ! g \colon A \subset \mathbb{R}^2 &\to \mathbb{R} \\
	(y, z) &\mapsto g(y, z) = x
\end{aligned}
\quad \text{tal que}
\quad
f(g(y, z), y, z) = 0 \quad \forall (y, z) \in \mathbb{R}^2
\]
Calculamos ahora $P_2(g, (1,-1))$ que es:
\[
P_2(g, a)(\phi) = g(a) + Jg(a)(\phi - a) + \frac{1}{2}(\phi - a)^tHg(a)(\phi-a) \quad a = (1,-1)
\]
Calculamos ahora $Jg(a)$:
\[
Jg(a) = -\left(\pdv{f}{x}(p) \right)^{-1} \begin{pmatrix} \pdv{f}{y}(p) & \pdv{f}{z}(p) \end{pmatrix}
=
-(1)^{-1} \begin{pmatrix} 0 & 1 \end{pmatrix} = \begin{pmatrix} 0 & -1 \end{pmatrix}
\]
Calculamos ahora $Hg(a)$. Para ello derivamos implicitamente, sabemos que $f(g(y, z), y, z) = 0$
y por lo tanto (Llamando $g(y, z) = x$):
\[
0 = \pdv{f(x, y, z)}{y} = x^2 + 2xy\pdv{x}{y} + e^x\pdv{x}{y}
\iff
\pdv{x}{y}\left( 2xy + e^x \right) + x^2 = 0
\]
Calculamos ahora $\pdv{^2x}{y \partial y}$, repitiendo el mismo proceso:
\[
0 = 2x\pdv{g}{y} +
\pdv{^2x}{y \partial y} \left( 2xy + e^x \right) + \pdv{x}{y} \left( 2x\pdv{x}{y} + 2x + e^x \pdv{x}{y} \right)
\]
De donde se obtiene (sustituyendo):
\[
\pdv{^2x}{y \partial y}(1,-1) = 0
\]
Ahora derivando parcialmente respecto a $z$ en la ecuacion anterior tenemos:
 \[
 0 = 2x\pdv{^2x}{z \partial y} + \pdv{^2x}{z \partial y} \left( 2xy +e^x \right) +
 \pdv{x}{y} \left( 2y \pdv{x}{z} + e^x \pdv{x}{z} \right)
 \implies
 \pdv{^2x}{z \partial y}(1,-1) = 0
 \]
 Repetimos ahora el proceso para obtener $\pdv{^2x}{z \partial z}$:
 \[
 0 = 2xy \pdv{x}{z} + e^x\pdv{x}{z} + 1
 \iff
 0 = \pdv{x}{z} \left(2xy + e^x \right) +1
 \]
 \[
 \pdv{^2x}{z \partial z} \left( 2xy + e^x \right) + \pdv{x}{z} \left( 2y\pdv{x}{z} + e^x \pdv{x}{z} \right) = 0
 \implies
 \pdv{^2x}{z \partial z}(1,-1) = -3
 \]
Y $Hg(a) = \begin{pmatrix} 0 & 0 \\ 0& -3 \end{pmatrix}$. Para concluir:
\[
P_2(g, a)(y, z) = 0+ \begin{pmatrix} 0 & -1 \end{pmatrix}
\left( \begin{pmatrix}y \\ z \end{pmatrix} - \begin{pmatrix} 1 \\ -1 \end{pmatrix} \right) + 
\frac{1}{2}\left( \begin{pmatrix} y & z \end{pmatrix} - \begin{pmatrix} 1 & -1 \end{pmatrix} \right)
\begin{pmatrix} 0 & 0 \\ 0 & -3 \end{pmatrix}
\left( \begin{pmatrix} y \\ z \end{pmatrix} - \begin{pmatrix} 1 \\ -1 \end{pmatrix} \right)
\]
\[
P_2(g, a)(y, z) = -3z^2 -4z+1
\]
 \newpage
 
 Primero calculamos $\pdv{F}{x}$ y $\pdv{F}{y}$:
 \[
 \pdv{F}{x}(x, y) = \pdv{}{x}h\left(\frac{y}{x}\right) = -h'\left(\frac{y}{x}\right)\frac{y}{x^2}
 \]
 \[
 \pdv{F}{y}(x,y) = \pdv{}{y}h\left(\frac{y}{x}\right) = \frac{h'\left(\frac{y}{x}\right)}{x}
 \]
 Y evidentemente la ecuacion se cumple:
 \[
 x \pdv{F}{x} + y \pdv{F}{y} = -x h'\left(\frac{y}{x}\right)\frac{y}{x^2} + y\frac{h'\left(\frac{y}{x}\right)}{x} = 0
 \]
 \\
 \\
 Ahora, definimos la function:
 \begin{align*}
G \colon A \subset \mathbb{R}^2 &\to \mathbb{R}^2 \\
(u, v) &\mapsto (x, y) = (u, uv)
\end{align*}
Y tenemos que:
\[
D(F \circ G)(x) = DF(G(x)) \circ DG(x)
\]
\[
DF(G(x)) =
\begin{pmatrix}
\pdv{F}{x}(u, uv) & \pdv{F}{y}(u, uv)
\end{pmatrix}
 \]
 \[
 DG(x) = 
 \begin{pmatrix}
1 & 0 \\
v & u
\end{pmatrix}
 \]
 Y por lo tanto:
 \[
 D(F \circ G)(u, v) = 
 \begin{pmatrix}
\pdv{F}{x}(u, uv) + \pdv{F}{y} (u, uv) v & \pdv{F}{y}(u, uv)u
\end{pmatrix}
 \]
 Observamos ahora que, por hipotesis:
 \[
 u \pdv{(F \circ G)}{u}(u, uv) = 0 \implies \pdv{(F \circ G)}{u}(u, uv) = 0 \quad \forall u, v \in A
 \]
 Y que por lo tanto $(F \circ G)_{\mid A}$ se puede expresar como:
 \[
 (F \circ G)_{\mid A}(u, v) = h(v) \implies (F \circ G \circ G^{-1})_{\mid A}(x,y) = h \left( G^{-1}_2(x, y) \right)
 \]
 Pero $G^{-1}(x, y) = \left(x, \frac{y}{x}\right)$ y por lo tanto:
 \[
 F_{\mid A} = h\left(\frac{y}{x}\right)
\]



\newpage
%======================
c) $ m = f(p)$ donde $p$ es el minimo calculado en el apartado b
\[
f\left( \frac{x}{\norm{x}} \right) = \frac{f(x)}{\norm{x}^2}
\implies f(x) = \norm{x}^2f \left( \frac{x}{\norm{x}} \right) \geq \norm{x}^2f(p) = \norm{x}^2m
\]
d) inmediato a partir de e \\
e)
\[
0 = \lim_{h \to 0} \frac{f(h) - f(0) - Df(0)(h)}{\norm{h}}
\iff
\lim_{h \to 0} \frac{f(h) - Df(0)(h)}{\norm{h}} = 0
\]
Pero
 \[
 \lim_{h \to 0} \frac{f(h)}{\norm{h}} = \lim_{h \to 0} f\left( \frac{h}{\sqrt{\norm{h}}} \right)
 =
 \lim_{h \to 0} f\left( \frac{h}{\norm{h}} \frac{\norm{h}}{\sqrt{\norm{h}}} \right)
 =
 \]
 \[
 =
  \lim_{h \to 0} f\left(\sqrt{\norm{h}} \frac{h}{\norm{h}}  \right)
  =
  f \left( \lim_{h \to 0} \sqrt{\norm{h}} \frac{h}{\norm{h}} \right)
  =
  f(0) = 0
 \]
 Por lo tanto, tomando $Df(0) = 0$, tenemos que:
 \[
 0 = \lim_{h \to 0} \frac{f(h) - f(0) - Df(0)(h)}{\norm{h}}
 \]
 Y $f$ es diferenciable en el origen
 
 \newpage
 %======================
 
 \[
 Jf(x, y) = \begin{pmatrix}
 -y^2e^{y-x^2}2x &
 2ye^{y-x^2} + y^2e^{y-x^2}
 \end{pmatrix}
 \]
 De la primera componente observamos que vale 0 $\iff \begin{cases} x =0 \\ y = 0 \end{cases}$
 \\
 La segunda, vale 0 $\iff \begin{cases} y = 0 \\ y = -2 \end{cases}$
 \\
 De donde obtenemos que los puntos criticos son $(0,-2)$ y $(x, 0) \quad \forall x \in \mathbb{R}$. Calculamos ahora $Hf(x, y)$:
 \[
 Hf(x, y) =
 \begin{pmatrix}
 4y^2x^2e^{y-x^2} - 2y^2e^{y-x^2} & -4xye^{y-x^2} - 2xy^2e^{y-x^2} \\
 -4xye^{y-x^2} -2xy^2e^{y-x^2} & 2e^{y-x^2} + 2ye^{y-x^2} + 2ye^{y-x^2} + y^2e^{y-x^2}
 \end{pmatrix}
\]
\[
Hf(0,-2) = \begin{pmatrix}
-8e^{-2} & 0 \\
0 & -2e^{-2}
\end{pmatrix}
\implies
f(0,-2) \text{ es maximo}
\]
\[
Hf(x, 0) =
\begin{pmatrix}
0 & 0 \\
0 & 2e^{-x^2}
\end{pmatrix}
\quad \text{que es semidefinida positiva}
 \]
 Para caracterizar los puntos $(x, 0)$ observamos que $f(x, y) \geq 0 \quad \forall (x, y) \in \mathbb{R}^2 $
 y que $f(x, 0) = 0 \quad \forall x \in \mathbb{R}^2$ y por lo tanto son minimos
\\
\\
\\
Los puntos criticos de $f$ en $K$ son $(0,0)$. Miramos ahora los puntos de la frontera $x = 0$.
Para ello definimos la funcion:
\[
F(x, y, \lambda) = y^2e^{y-x^2} + \lambda x
\]
\[
JF(x, y, \lambda) =
\begin{pmatrix} -2xy^2e^{y-x^2} + \lambda & 2ye^{y-x^2} + y^2e^{y-x^2} & x \end{pmatrix}
\]
\[
JF(x, y, \lambda) = 0 \iff
\begin{cases}
-2xy^2e^{y-x^2} + \lambda = 0 \\
 2ye^{y-x^2} + y^2e^{y-x^2} = 0 \\
 x = 0
\end{cases}
\implies
x = 0, \lambda = 0, y = \begin{cases} -2 \\ 0 \end{cases}
\]
Ahora la frontera $y = 1$:
\[
F(x, y, \lambda) = y^2e^{y-x^2} + \lambda (y-1)
\implies
JF(x, y, \lambda) =
\begin{pmatrix} -2xy^2e^{y-x^2} & 2ye^{y-x^2} + y^2e^{y-x^2} + \lambda & y - 1 \end{pmatrix}
\]
\[
JF(x, y, \lambda) = 0 \iff x = 0, y = 1, \lambda = -3e
\]
Con la frontera $x^2 = y$:
\[
F(x, y, \lambda) = y^2e^{y-x^2} + \lambda (x^2 - y)
\implies
JF(x, y, \lambda) =
\begin{pmatrix} -2xy^2e^{y-x^2} + 2\lambda x & 2ye^{y-x^2} + y^2e^{y-x^2} - \lambda &
x^2 - y \end{pmatrix}
\]
\[
JF(x, y, \lambda) = 0 \iff x = 0, y = 0, \lambda = 0
\]
\newpage
Ahora con la restriccion $x=0$ y $y=1$:
\[
F(x, y, \lambda_1, \lambda_2) = y^2e^{y-x^2} + \lambda_1 x + \lambda_2 (y-1) \implies
\]
\[
\implies JF(x, y, \lambda_1, \lambda_2) = 
\begin{pmatrix}
-2xy^2e^{y-x^2} + \lambda_1 & 2ye^{y-x^2} + y^2e^{y-x^2} + \lambda_2 & x & y
\end{pmatrix}
\]
\[
JF(x, y, \lambda_1, \lambda_2) = 0 \iff x=0,y=0,\lambda_1=0,\lambda_2=0
\]
Ahora con la restriccion $x=0$ y $x^2=y$:
\[
F(x, y, \lambda_1, \lambda_2) = y^2e^{y-x^2} + \lambda_1 x + \lambda_2 (x^2 - y) \implies
\]
\[
\implies JF(x, y, \lambda_1, \lambda_2) = 
\begin{pmatrix}
-2xy^2e^{y-x^2} + \lambda_1 + 2\lambda_2x & 2ye^{y-x^2} + y^2e^{y-x^2} - \lambda_2 & x & x^2 - y
\end{pmatrix}
\]
\[
JF(x, y, \lambda_1, \lambda_2) = 0 \iff x=0,y=0,\lambda_1=0,\lambda_2=0
\]
Ahora con $y=1$ y $x^2=y$:
\[
F(x, y, \lambda_1, \lambda_2) = y^2e^{y-x^2} + \lambda_1 (y-1) + \lambda_2 (x^2 - y) \implies
\]
\[
\implies JF(x, y, \lambda_1, \lambda_2) = 
\begin{pmatrix}
-2xy^2e^{y-x^2} + 2\lambda_2x & 2ye^{y-x^2} + y^2e^{y-x^2} +\lambda_1 - \lambda_2 & y-1 & x^2 - y
\end{pmatrix}
\]
\[
JF(x, y, \lambda_1, \lambda_2) = 0 \iff x=\pm 1, y =1, \lambda_1 = 1-3e, \lambda_2=1
\]
Por ultimo, tomamos $x=0$, $y=1$, $x^2=y$:
\[
F(x, y, \lambda_1, \lambda_2, \lambda_3) = y^2e^{y-x^2} + \lambda_1 x + \lambda_2 (y-1) +
\lambda_3 (x^2 - y) \implies
\]
\[
\implies JF(x, y, \lambda_1, \lambda_2, \lambda_3) = 
\begin{pmatrix}
-2xy^2e^{y-x^2} + \lambda_1 + 2\lambda_2 x & 2ye^{y-x^2} + y^2e^{y-x^2} + \lambda_2 - \lambda_3
& x & y-1 & x^2 - y
\end{pmatrix}
\]
\[
JF(x, y, \lambda_1, \lambda_2, \lambda_3) \neq 0 \quad
\forall (x, y, \lambda_1, \lambda_2, \lambda_3) \in \mathbb{R}^5
\]
Los candidatos a maximo y aminimo son:
\[
\begin{cases}
(0,0) \\
(0,1) \\
(1, 1)
\end{cases}
\implies
\begin{cases}
f(0, 0) = 0 \\
f(0, 1) = e \\
f(1, 1) = 1
\end{cases}
\implies
\begin{cases}
\text{maximo: } (0, 1) \\
\text{minimo: } (0, 0)
\end{cases}
\]

%==================
\newpage

La funcion de Lagrange es $F(x, y, z, \lambda) = x + y + z + \lambda(x^2 + y^2 - 2z^2 - 6)$
\[
JF(x, y, z, \lambda) = \begin{pmatrix}
1 + 2x\lambda & 1 + 2y\lambda & 1 - 4z\lambda & x^2 + y^2 -2z^2 - 6
\end{pmatrix}
\]
\[
JF(x, y, z, \lambda) = 0 \iff
\begin{cases}
(x, y, z, \lambda) = \left(2,2,-1,\frac{1}{4} \right) \\
(x, y, z, \lambda) = \left(-2,-2,1,\frac{-1}{4} \right) \\
\end{cases}
\]
Primero calculamos $T_{(2,2,-1)}M$ y $T_{(-2,-2,1)}M$, que en este caso es:
\[
T_{(2,2,-1)}M = Nuc(DF(2,2,-1)) = Nuc \begin{pmatrix}
 1 & 1 & 1
\end{pmatrix} = \left\{ (x, y, z) \in \mathbb{R}^3 \mid x + y + z = 0\right\}
\]
\[
T_{(-2,-2,-1)}M = Nuc(DF(-2,-2,1)) = Nuc \begin{pmatrix}
 1 & 1 & 1
\end{pmatrix} = \left\{ (x, y, z) \in \mathbb{R}^3 \mid x + y + z = 0 \right\}
\]
Y $HF(x,  y, z, \lambda)$:
\[
HF(x, y, z, \lambda) = \begin{pmatrix}
2\lambda & 0 & 0 & 2x \\
0 & 2\lambda & 0 & 2y \\
0 & 0 & -4\lambda & -4z \\
2x & 2y & -4z & 0
\end{pmatrix}
\]
\\ \\ \\ \\
$f$ es de clase $C^\infty$ y $det\left( \pdv{f}{z} \right) \neq 0$ y por lo tanto existe $h(x, y)$
de clase $C^\infty$ y (llamando $z = h(x, y)$):
\[
1 + \pdv{z}{x} = 0 \iff \pdv{z}{x} = -1
\]
\[
1 + \pdv{z}{y} = 0 \iff \pdv{z}{y} = -1
\]
\end{document}