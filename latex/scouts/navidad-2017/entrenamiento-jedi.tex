\documentclass[72pt]{article}
\usepackage[utf8]{inputenc}

\usepackage{geometry}
\geometry{a4paper}

\usepackage{fancyhdr}
\usepackage{calc}
\usepackage{multirow}
\usepackage[explicit]{titlesec}
\usepackage{graphicx}
\usepackage{xcolor}
\usepackage[hidelinks]{hyperref}

\newlength{\w}
\setlength{\w}{\textwidth / 5 - 1.18pt}

\newcommand*\getlength[1]{\number#1}

\pagestyle{fancy}
\fancyhf{}
\rhead{\textit{Grupo Scout 109 Monte Clavijo} \\}
\lhead{Campamento de Navidad 2017 \\ Nieva de Cameros (La Rioja)} %CAMPAMENTO Y LUGAR
\cfoot{\includegraphics[width=\getlength{\w}sp]{logo}
	\includegraphics[width=\getlength{\w}sp]{logo}\includegraphics[width=\getlength{\w}sp]{logo}
	\includegraphics[width=\getlength{\w}sp]{logo}\includegraphics[width=\getlength{\w}sp]{logo}}

\titleformat{\section}
  {\Large\bfseries}{\thesection}{5em}{#1}[{\titlerule}]

\begin{document}
\large
\fontencoding{T1}\fontfamily{lmss}\fontseries{m}\fontshape{n}\selectfont
\begin{tabular}{|p{2cm}|c|}
\hline
\multirow{5}{*}{\includegraphics[width=2cm]{logo}} & \textbf{CLAN} \\ % RAMA
\cline{2-2}
& \textbf{ENTRENAMIENTO JEDI} \\ %TITULO
\cline{2-2}
& \parbox{\textwidth-4cm}{Fecha y momento del dia: \textbf{27 diciembre - tarde}} \\ %Fecha
& \parbox{\textwidth-4cm}{Duracion: \textbf{1.5 horas}} \\ %Duracion
& \parbox{\textwidth-4cm}{Destinatarios/as: \textbf{jovenes de 17 a 21 años}} \\ %Destinatarios
\hline
\end{tabular}

\section*{DESCRIPCI\'ON} 
\textit{“¿Preparado estás? ¿Qué sabes tú de estar preparado? Durante
ochocientos años he entrenado a los Jedi. Yo decidiré quién debe ser
entrenado.”}

\section*{OBJETIVOS}
\begin{itemize}
    \item Pasar un rato distendido entre todo el Clan.
    \item Desarrollar la competitividad de una manera sana.
    \item Y, sobre todo, ¡¡SER EL MEJOR JEDI!!
\end{itemize}

\section*{DESARROLLO}
Los 18 rovers se dividen en dos equipos (9 en cada equipo) y a cada uno
se le asigna un equipo: equipo rojo y equipo azul.
El juego consistirá en un entrenamiento para convertise en auténticos
Jedi. Se dividirá el conjunto de la actividad en diferentes pruebas en las
que se juzgará la capacidad de los padawan para controlar la fuerza que
les trasmiten los midiclorianos.
\begin{enumerate}
    \item Construir el sable laser: todo buen jedi debe tener un sable de luz
        que le permita luchar contra las tropas de asalto del ejército
        imperial. Para ello cogeremos un churro de piscina y lo partiremos
        por la mitad. En la empuñadura pondremos cinta americana y le
        haremos los adornos con cinta aislante negra.
    \item Aprender a usar el sable: todo buen jedi ha de tener rápidos reflejos
        para parar los disparos laser del enemigo. En nuestro entrenamiento
        usaremos pelotas de tenis y un bate de beisbol. Se lanzarán un
        número determinado de pelotas a cada persona, los lanzadores
        serán los del equipo contrario. Aquel equipo que más consiga batear
        bien será el ganador de esta prueba.
    \item “Hazlo o no lo hagas, pero no lo intentes”: levitar objetos puede ser
        una tarea harto complicada, pero si quieres controlar la fuerza no te
        queda más remedio que saber hacerlo bien. Esta prueba consistirá
        en una carrera de relevos una cuchara con un huevo sujetada con la boca.
    \item Conquistar la base enemiga: esta será la última prueba para que
        uno de los dos grupos de padawans se conviertan en auténticos
        jedis. Cada uno de los equipos tendrá una base y su objetivo será
        tomar la base enemiga. Esto se conseguirá haciendo que todos los
        componentes del equipo estén en ella a la vez. Los padawan podrán
        atacar y defender usando sus sables de luz, si se golpea a un
        contrincante en la cabeza este tendrá que quedarse sentado en el
        suelo hasta que alguien de su equipo le toque.
\end{enumerate}

\section*{MATERIALES}
\begin{itemize}
    \item 5 churros de piscina azules
    \item 5 churros de piscina rojos
    \item Cinta americana
    \item Cinta aislante negra
    \item Bate de beisbol
    \item Pelotas
    \item huevos
    \item cucharas
\end{itemize} 

\section*{SUGERENCIAS}
\begin{itemize}
    \item Hablar como Yoda
\end{itemize}

\section*{EVALUACION} 
\begin{itemize}
    \item ¿Han participado todos?
    \item ¿Nos lo hemos pasado bien, respetando en todo momento a los compañeros?
    \item ¿Durante la actividad han imperado los valores presentes en la ley y la
        promesa scout?
\end{itemize}

\section*{FUENT E}
G. S. 109 Monte Clavijo
\end{document}
