\documentclass[72pt]{article}
\usepackage[utf8]{inputenc}

\usepackage{geometry}
\geometry{a4paper}

\usepackage{fancyhdr}
\usepackage{calc}
\usepackage{multirow}
\usepackage[explicit]{titlesec}
\usepackage{graphicx}
\usepackage{xcolor}
\usepackage[hidelinks]{hyperref}

\newlength{\w}
\setlength{\w}{\textwidth / 5 - 1.18pt}

\newcommand*\getlength[1]{\number#1}

\pagestyle{fancy}
\fancyhf{}
\rhead{\textit{Grupo Scout 109 Monte Clavijo} \\}
\lhead{Campamento de Navidad 2017 \\ Nieva de Cameros (La Rioja)} %CAMPAMENTO Y LUGAR
\cfoot{\includegraphics[width=\getlength{\w}sp]{logo}
	\includegraphics[width=\getlength{\w}sp]{logo}\includegraphics[width=\getlength{\w}sp]{logo}
	\includegraphics[width=\getlength{\w}sp]{logo}\includegraphics[width=\getlength{\w}sp]{logo}}

\titleformat{\section}
  {\Large\bfseries}{\thesection}{5em}{#1}[{\titlerule}]

\begin{document}
\large
\fontencoding{T1}\fontfamily{lmss}\fontseries{m}\fontshape{n}\selectfont
\begin{tabular}{|p{2cm}|c|}
\hline
\multirow{5}{*}{\includegraphics[width=2cm]{logo}} & \textbf{CLAN} \\ % RAMA
\cline{2-2}
& \textbf{GRAND PRIX SCOUT} \\ %TITULO
\cline{2-2}
& \parbox{\textwidth-4cm}{Fecha y momento del dia: \textbf{27 diciembre - mañana}} \\ %Fecha
& \parbox{\textwidth-4cm}{Duracion: \textbf{1.5 horas}} \\ %Duracion
& \parbox{\textwidth-4cm}{Destinatarios/as: \textbf{jovenes de 17 a 21 años}} \\ %Destinatarios
\hline
\end{tabular}

\section*{DESCRIPCI\'ON}
Esta actividad está basada en el juego del “Grand Prix” en la que dos
equipos a la vez competirán por obtener la mayor puntuación posible y
hacerse vencedor a través de una serie de pruebas diferentes. Cada
prueba es un tipo de destreza (fuerza, puntería, inteligencia...)
\section*{OBJETIVOS}
\begin{itemize}
    \item Pasar un rato divertido y agradable todos juntos
	\item Reírse mucho
	\item Participar todos en cada actividad
	\item Y tener una actitud positiva que permita el desarrollo de la actividad
\end{itemize}

\section*{DESARROLLO}

Los 18 rovers se dividen en dos equipos (9 en cada equipo) y a cada uno
se le asigna un equipo: equipo rojo y equipo azul, cada miembro del
equipo deberá llevar pegado en la frente un gomet con su respectivo
color. El juego se divide en seis pruebas diferentes:
\begin{enumerate}
	\item \textbf{Fuerza:}
		Se hace un círculo en el suelo con una tiza y un miembro de un
		equipo y otro del otro equipo tienen que luchar (empujándose,
		cogiéndose) hasta que uno de los dos salga del círculo, para luchar
		pueden usar la esterilla (enroscada). A la hora de elegir que
		miembro del equipo sale cada equipo elige al suyo sin que el otro se
		entere así no hay pique de es más fuerte este u el otro. Gana el
		equipo que más personas haya sacado fuera.
	\item \textbf{Cálculo:}
		Consiste en adivinar la solución de la operación matemática que se
		diga y tirarse encima de la solución (estará puesto con un papel
		encima de una esterilla). Cada vez sale un jugador de cada equipo
		pero a resolver la cuenta participa todo el equipo. Gana el equipo
		que acierte o en el caso de acertarla los dos equipos, el jugador que
		este debajo. Esta prueba la gana el equipo con más puntuación.
	\item \textbf{inteligencia:}
		Se hacen preguntas y para contestarlas una vez sabida la respuesta,
		un miembro de cada equipo (diferente cada viaje) debe ir a por la
		pañoleta y llevarla hasta un punto, el primero que deje la pañoleta
		en su lugar hace que su equipo responda primero la pregunta, solo
		hay una oportunidad de respuesta por pregunta. Gana el equipo que
		más preguntas se haya sabido.
	\item \textbf{Maña:}
		Se ata una media con una pelota de tenis dentro en el enganche del
		pantalón, en el caso de no haber enganche, se atará con
		imperdibles. Un jugador de cada equipo a la vez al oír “1,2,3 ya”
		deberán golpear otra pelota con un solo movimiento (de atrás a
		adelante). Gana el jugador que más lejos envíe la pelota. Y gana el
		equipo con mayor puntuación.
		También habrá otra prueba en la que uno de un equipo se pone un
		globo en el culo, y el del otro equipo tiene que intentar romperlo. Se
		explicará el día de la actividad.
	\item \textbf{Compañerismo y competitividad:}
		Al ser un juego de equipo participa todo el equipo a la vez, cada
		equipo solo puede tener de pie a 5 personas, en total 10 pies, los
		demás que se las ingenien como puedan. Cada equipo tiene que
		intentar tirar a los miembros del otro equipo. Cuando se hayan
		tirado a las personas que no tienen pies en el suelo tendrán que
		tiran al suelo a las personas que están de pie. El equipo con más
		personas en pie gana.
	\item \textbf{Puntería:}
		Se divide en dos cada equipo, la mitad del equipo se tiene que
		poner un gorro de plástico y una bolsa de basura. Se le manchará el
		gorro con nata, la mitad restante del grupo desde una distancia
		tendrá que lanzar gusanitos en la cabeza de su compañero y que se
		queden pegados. Gana el equipo que más gusanitos haya pegado.
		Tras finalizar está prueba se termina el grand prix pueden chupar la
		nata de la cabeza y comer gusanitos.
\end{enumerate}

\section*{MATERIALES}
\begin{itemize}
    \item Gusanitos
    \item Celo
	\item tiza
	\item Preguntas
	\item 1 paquete de globos
	\item Medias (4)
	\item Imperdibles
	\item Pelotas/naranjas
	\item Gusanitos y nata
	\item Gorros de plástico y bolsas de basura
	\item Galletas (dos paquetes)
\end{itemize}
\section*{SUGERENCIAS}

\begin{itemize}
    \item
\end{itemize}

\section*{EVALUACION}
\begin{itemize}
    \item ¿Han participado todos?
	\item ¿Nos lo hemos pasado bien, respetando en todo momento a los
compañeros?
\end{itemize}

\section*{FUENTE}
G. S. 109 Monte Clavijo
\end{document}
