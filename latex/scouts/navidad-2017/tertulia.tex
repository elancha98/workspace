\documentclass[72pt]{article}
\usepackage[utf8]{inputenc}

\usepackage{geometry}
\geometry{a4paper}

\usepackage{fancyhdr}
\usepackage{calc}
\usepackage{multirow}
\usepackage[explicit]{titlesec}
\usepackage{graphicx}
\usepackage{xcolor}
\usepackage[hidelinks]{hyperref}

\newlength{\w}
\setlength{\w}{\textwidth / 5 - 1.18pt}

\newcommand*\getlength[1]{\number#1}

\pagestyle{fancy}
\fancyhf{}
\rhead{\textit{Grupo Scout 109 Monte Clavijo} \\}
\lhead{Campamento de Navidad 2017 \\ Nieva de Cameros (La Rioja)} %CAMPAMENTO Y LUGAR
\cfoot{\includegraphics[width=\getlength{\w}sp]{logo}
	\includegraphics[width=\getlength{\w}sp]{logo}\includegraphics[width=\getlength{\w}sp]{logo}
	\includegraphics[width=\getlength{\w}sp]{logo}\includegraphics[width=\getlength{\w}sp]{logo}}

\titleformat{\section}
  {\Large\bfseries}{\thesection}{5em}{#1}[{\titlerule}]

\begin{document}
\large
\fontencoding{T1}\fontfamily{lmss}\fontseries{m}\fontshape{n}\selectfont
\begin{tabular}{|p{2cm}|c|}
\hline
\multirow{5}{*}{\includegraphics[width=2cm]{logo}} & \textbf{CLAN} \\ % RAMA
\cline{2-2}
& \textbf{TERTULIA} \\ %TITULO
\cline{2-2}
& \parbox{\textwidth-4cm}{Fecha y momento del dia: \textbf{29 diciembre - tarde}} \\ %Fecha
& \parbox{\textwidth-4cm}{Duracion: \textbf{1.5 horas}} \\ %Duracion
& \parbox{\textwidth-4cm}{Destinatarios/as: \textbf{jovenes de 17 a 21 años}} \\ %Destinatarios
\hline
\end{tabular}

\section*{DESCRIPCI\'ON} 
Esta actividad va a consistir en un tertulia sobre cotilleos y su veracidad o
falsedad con relación a la temática de la acampada: la televisión.

\section*{OBJETIVOS}
\begin{itemize}
    \item Dialogar todos juntos.
    \item Conocer un poco más a todas las personas.
    \item Desarrollar la astucia.
\end{itemize}

\section*{DESARROLLO}
Nos sentaremos en una mesa, con bolis y papel, cada persona escribirá
varios cotilleos (mínimo 2) siendo algunos falsos y otros verdaderos,
posteriormente se introducirán todos en una caja y luego se sacarán y se
irán leyendo uno por uno, dialogándose entre todos si se cree que son
verdad o no, con la decisión final se preguntará al polígrafo(persona o
personas del cotilleo) por la respuesta final y se verá si hemos acertado.

\section*{MATERIALES} 
\begin{itemize}
    \item Papel
    \item Boli
    \item Caja
\end{itemize}

\section*{SUGERENCIAS}
\begin{itemize}
    \item Tomarse la actividad con humor.
\end{itemize}

\section*{EVALUACION} 
\begin{itemize}
    \item ¿Han participado adecuadamente?
    \item ¿Han surgido problemas?
\end{itemize}

\section*{FUENTE}
G. S. 109 Monte Clavijo
\end{document}
