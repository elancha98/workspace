\documentclass[72pt]{article}
\usepackage[utf8]{inputenc}

\usepackage{geometry}
\geometry{a4paper}

\usepackage{fancyhdr}
\usepackage{calc}
\usepackage{multirow}
\usepackage[explicit]{titlesec}
\usepackage{graphicx}
\usepackage{xcolor}
\usepackage[hidelinks]{hyperref}

\newlength{\w}
\setlength{\w}{\textwidth / 5 - 1.18pt}

\newcommand*\getlength[1]{\number#1}

\pagestyle{fancy}
\fancyhf{}
\rhead{\textit{Grupo Scout 109 Monte Clavijo} \\}
\lhead{Campamento de Navidad 2017 \\ Nieva de Cameros (La Rioja)} %CAMPAMENTO Y LUGAR
\cfoot{\includegraphics[width=\getlength{\w}sp]{logo}
	\includegraphics[width=\getlength{\w}sp]{logo}\includegraphics[width=\getlength{\w}sp]{logo}
	\includegraphics[width=\getlength{\w}sp]{logo}\includegraphics[width=\getlength{\w}sp]{logo}}

\titleformat{\section}
  {\Large\bfseries}{\thesection}{5em}{#1}[{\titlerule}]

\begin{document}
\large
\fontencoding{T1}\fontfamily{lmss}\fontseries{m}\fontshape{n}\selectfont
\begin{tabular}{|p{2cm}|c|}
\hline
\multirow{5}{*}{\includegraphics[width=2cm]{logo}} & \textbf{CLAN} \\ % RAMA
\cline{2-2}
& \textbf{PASAPALABRA} \\ %TITULO
\cline{2-2}
& \parbox{\textwidth-4cm}{Fecha y momento del dia: \textbf{27 diciembre - tarde}} \\ %Fecha
& \parbox{\textwidth-4cm}{Duracion: \textbf{1.5 horas}} \\ %Duracion
& \parbox{\textwidth-4cm}{Destinatarios/as: \textbf{jovenes de 17 a 21 años}} \\ %Destinatarios
\hline
\end{tabular}

\section*{DESCRIPCI\'ON}
    Jugaremos al conocido juego ``Pasapalabra'', haciendo una competición por equipos.
\section*{OBJETIVOS} 
\begin{itemize}
    \item Pasarnoslo muy requetebien
    \item Aprender nuevo vocabulario
\end{itemize}

\section*{DESARROLLO} 
    Jugaremos al ``fantabuloso'' juego de pasapalabra. Primero, dividiremos al clan en
    grupos, el número es irrelevante, pero se recomiendan equipos de 3 o 4 personas
    como máximo. A partir de aquí, usaremos la app de ordenador para jugar. Cada equipo
    tendrá un ``rosco'' con letras y un tiempo inicial de 200 segundos. Y se procederá
    de forma normal, con las normas originales.
\section*{MATERIALES}
\begin{itemize} 
    \item Ordenador
    \item Proyector
    \item Móvil con anclaje por USB
    \item App de Pasapalabra
\end{itemize}
\section*{SUGERENCIAS} 

\begin{itemize}
    \item No hacer los equipos muy grandes.
\end{itemize}

\section*{EVALUACION} 
\begin{itemize}
    \item ¿Se lo han pasado bien?
    \item ¿Han aprendido nuevas palabras?
    \item ¿Han respetado los turnos de palabra?
\end{itemize}

\section*{FUENTE} 
G. S. 109 Monte Clavijo
\end{document}
