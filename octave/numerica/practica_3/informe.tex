\documentclass[12pt]{article}
\usepackage[utf8]{inputenc}
\usepackage[paper=a4paper,margin=20mm]{geometry}
\usepackage{amsmath,amsfonts, amssymb}
\usepackage[hidelinks]{hyperref}

\hypersetup{
colorlinks,
linkcolor={red}
}

\begin{document}
\title{Informe de práctica: Mínimos quadrados y descomposición QR}
\author{Ernesto Lanchares y Miquel Ortega}
\date{}
\maketitle

\setlength\parindent{0pt}

La práctica llevada a cabo hace uso del siguiente resultado:


\begin{itemize}
\item 
Sean $x = (x_0,\dots,x_n)^T$ e $y = (y_0,\dots,y_n)^T$ vectores pertenecientes a $\mathbb{R}^{N+1}$. Un polinomio $p_M(x)$ cumple
\begin{equation}\label{eq:1}
\sqrt{\sum_{i=0}^{N} (p_M(x_i) - y_i)^2)} 
\leq \sqrt{\sum_{i=0}^{N} (q_M(x_i) - y_i)^2)}
\end{equation}
para todo $q_M(x) \in R_M[x]$ si y solo si sus coeficientes $a_0, a_1,\dots,a_M$ son solución de las ecuaciones normales.
\end{itemize}

Procedamos a la demostración. Sea $A\in\mathcal{M}_{N+1, M+1}$ la matriz donde $A_{i,j} = x_i^{(M+1-j)}$ y $a = (a_0, a_1,...,a_n)^T$. Entonces, las ecuaciones normales consisten en imponer $A^TAa = A^Ty$. Esto se puede reescribir como 
\[A^Ta-A^Ty = 0\]
\[Aa(y-b)^T\]



\end{document}
\grid
\grid
